% !TeX encoding = UTF-8
% !TeX spellcheck = it_IT
% !TeX root = tabellequarto.tex

\chapter{La funzione esponenziale}
\label{sec:LaFunzioneEsponenziale}
\section{Proprietà delle potenze}
\label{subsec:ProprietaDellePotenze}
Iniziamo con ripassare le proprietà delle potenze. La tabella\nobs\vref{tab:potenzeriepilogo} ne elenca le principali proprietà. Per le potenze a esponente reale la base deve un numero reale positivo. Se questo non viene rispettato, possiamo avere delle situazioni paradossali come la seguente:
\begin{cesempiot}{Base negativa}{}
\[-2=\sqrt[3]{-8}=(-8)^{\frac{1}{3}}=(-8)^{\frac{2}{6}}=\sqrt[6]{(-8)^{2}}=\sqrt[6]{64}=2\]
\end{cesempiot}
 \begin{table}
	\centering
	\begin{tabular}{RCLL}
		\toprule
		a^n&=&\overbrace{a\times a\times\cdots\times a}^{n{}\mbox{volte}}&\forall a\in\R\qquad n\in\Ni\qquad n>1 \\[.6cm]
		a^0&=&1&\forall a\in\R\qquad a\neq 0\\[.6cm]
		0^0&=&?\\[.6cm]
		a^1&=&a&\forall a\in\R\\[.6cm]
		
		a^n\cdot a^m&=&a^{n+m}&\forall a\in\R\qquad n,m\in\Ni\\[.6cm]
		a^m\div a^n&=&a^{m-n}&\forall a\in\R\qquad a\neq 0\qquad n,m\in\Ni\\[.6cm]
		\left(a^n\right)^m&=&a^{n\cdot m}&\forall a\in\R\\[.6cm]
		a^{n}b^{n}&=&{\left(ab\right)}^n& a,b\in\R\qquad n\in\Ni\\[.6cm]
		a^{n}\div b^{n}&=&{\left(a\div  b\right)}^n& a,b\in\R\qquad b\neq 0\qquad n\in\Ni\\[.6cm]
		\left(\dfrac{a}{b}\right)^n&=&\dfrac{a^n}{b^n}& a,b\in\R\qquad b\neq 0\qquad n\in\Ni\\[.6cm]
		a^{-n}&=&\left(\dfrac{1}{a}\right)^n&\forall a\in\R\qquad a\neq 0\qquad n\in\Ni \\[.6cm]
		a^{\frac{m}{n}}&=&\sqrt[n]{a^{m}}&\forall a\in\R\qquad a\geq 0\qquad n,m\in\Ni \\[.6cm]
		\bottomrule
	\end{tabular}
	\caption{Proprietà delle potenze}
	\label{tab:potenzeriepilogo}
\end{table}
\begin{definizionet}{Funzione esponenziale}{}
	Chiamo funzione esponenziale\index{Funzione!Esponenziale} una funzione del tipo \[y=a^x\quad
	\text{per $a>0$.} \]
\end{definizionet}
Il comportamento della funzione esponenziale varia al variare della base. La figura\nobs\vref{fig:funzioniEsempio6} mostra il comportamento per $a>1$.
\begin{enumerate}
\item Il grafico della funzione esponenziale occupa il semipiano positivo delle y.
\item Tutti i grafici passano per il punto $(0,1)$.
\item La funzione è crescente\index{Funzione!Crescente} all'aumentare dell'incognita. $x_1<x_2\quad f(x_1)<f(x_2)$ 
\item L'asse delle $x$ è un asintoto orizzontale\index{Asintoto!orizzontale}.
\end{enumerate}
 La figura\nobs\vref{fig:funzioniEsempio7} mostra il comportamento della funzione esponenziale per $0<a<1$. 
 \begin{enumerate}
 \item Il grafico della funzione esponenziale occupa il semipiano positivo delle y.
 \item Tutti i grafici passano per il punto $(0,1)$.
 \item La funzione è  decrescente\index{Funzione!Decrescente} all'aumentare dell'incognita. $x_1<x_2\quad f(x_1)>f(x_2)$ 
 \item L'asse delle $x$ è un asintoto orizzontale\index{Asintoto!orizzontale}.
 \end{enumerate}
\begin{figure}
\centering
%\begin{center}
\begin{tikzpicture}[>=triangle 90]
\begin{axis}
[xmin=-6,xmax=6,ymin=-1,ymax=6, %grid,
axis x line=middle,xtick={-6,-5,...,6},ytick={-6,-5,...,6},
axis y line=middle,xlabel=$x$,ylabel=$y$]
\addplot[samples=300] {(0.5^(x))};
\addplot [samples=300] {(2^(x))};

\end{axis}
\node (a1) at (1,1.5) {$a>1$};
\node (a2) at (6,1.5) {$0<a<1$};
\end{tikzpicture}
\caption{Funzione esponenziale riepilogo}
\label{fig:FunzioneExp}
%\end{center}
\end{figure}
\begin{figure}
	\centering
	\begin{subfigure}[b]{.4\linewidth}
		\centering
		\includestandalone[width=\textwidth]{quarto/funzioniBase/esempio6}
		\caption{$a>1$}
		\label{fig:funzioniEsempio6}
	\end{subfigure}\qquad
	\centering
	\begin{subfigure}[b]{.4\linewidth}
		\centering
		\includestandalone[width=\textwidth]{quarto/funzioniBase/esempio7}
		\caption{$0<a<1$}
		\label{fig:funzioniEsempio7}
	\end{subfigure}%
		\caption{Funzione esponenziale}
		\label{fig:funzExp2}
\end{figure}
\section{Logaritmo}
\label{sec:Lograritmo}
%Un'equazione esponenziale non ha sempre una soluzione per esempio \[10^x=20\] In questo caso la soluzione esiste ma non è nota, chiamo logaritmo\index{Logaritmo} di venti in base dieci l'esponente che bisogna dare a dieci per ottenere venti e lo indico con $\log_{10}20$
\begin{definizionet}{Logaritmo}{}
	Chiamo logaritmo in base $a$ di $b$ l'esponente che bisogna dare ad $a$ per ottenere $b$ \[x=\log_ba\Leftrightarrow b^{x}=a\quad a>0\quad a\neq  1\quad b>0 \]
\end{definizionet} 
\section{Funzione Logaritmica}
\label{FunzioneLogaritmica}
La funzione $y=log_ax$ si chiama funzione logaritmica\index{Funzione!Logaritmica} è fortemente collegata alla funzione esponenziale $y=a^x$. La figura\nobs\vref{fig:funzioniLogEsempio1} confronta la funzione esponenziale e la funzione logaritmica  quando la base è maggiore di uno $a>1$. La curva è simmetrica rispetto alla bisettrice del primo terzo quadrante. 
\begin{enumerate}
	\item Il grafico della funzione logaritmo occupa il semipiano positivo delle x.
	\item Tutti i grafici passano per il punto $(1,0)$.
	\item La funzione è crescente\index{Funzione!Crescente} all'aumentare dell'incognita. $x_1<x_2\quad f(x_1)<f(x_2)$ 
	\item L'asse delle $y$ è un asintoto verticale\index{Asintoto!verticale}.
\end{enumerate} 
\begin{figure}
	\centering
	\begin{subfigure}[b]{.4\linewidth}
		\centering
		\includestandalone[width=\textwidth]{quarto/FunzioniLog/esempio1}
		\caption{$a>1$}
		\label{fig:funzioniLogEsempio1}
	\end{subfigure}\qquad
	\centering
	\begin{subfigure}[b]{.4\linewidth}
		\centering
		\includestandalone[width=\textwidth]{quarto/FunzioniLog/esempio2}
		\caption{$0<a<1$}
		\label{fig:funzioniLogEsempio2}
	\end{subfigure}%
	\caption{Funzioni esponenziali e logaritmiche}
	\label{fig:funzExp1}
\end{figure}
La figura\nobs\vref{fig:funzioniLogEsempio2} confronta la funzione esponenziale e la funzione logaritmica  quando la base è compresa fra zero e uno $0<a<1$ . La curva è simmetrica rispetto alla bisettrice del primo terzo quadrante. 
\begin{enumerate}
	\item Il grafico della funzione logaritmo occupa il semipiano positivo delle x.
	\item Tutti i grafici passano per il punto $(1,0)$.
	\item La funzione è decrescente\index{Funzione!Decrescente} all'aumentare dell'incognita. $x_1<x_2\quad f(x_1)>f(x_2)$ 
	\item L'asse delle $y$ è un asintoto verticale\index{Asintoto!verticale}.
\end{enumerate}
La figura\nobs\vref{fig:FunzioniLogEsempio3} confranta fre di loro i due grafici.
\begin{figure}
	\centering
	\includestandalone[width=0.6\linewidth]{quarto/FunzioniLog/esempio3}
	%\includegraphics[width=0.7\linewidth]{./}
	\caption{Funzioni logaritmiche}
	\label{fig:FunzioniLogEsempio3}
\end{figure}
\section{Razionale fratta}
\begin{definizionet}{Funzione razionale fratta}{}
	Chiamo funzione razionale\index{Funzione!Razionale fratta} una funzione del tipo \[y=\dfrac{A(x)}{B(x)}\] con $A(x)$ e $B(x)$ due polinomi.
\end{definizionet}
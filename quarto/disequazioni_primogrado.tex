\chapter{Disequazioni di primo grado}
\label{cha:DisequazioniDiPrimogrado}
 \section{Diseguaglianze}
\label{sec:Disequglianze}
Iniziamo con un po' di vocabolario. La tabella~\vref{tab:disuguaglianze} mostra le possibili disuguaglianze\index{Disuguaglianza} e il modo corretto di  leggerle.
\begin{table}
\centering
\begin{tabular}{lcll}
	\toprule
<&$a<b$&minore stretto&<<a è minore di b>>\\
>&$a>b$&maggiore stretto& <<a è maggiore di b>>\\
$\leq$&$a\leq b$&minore o uguale& <<a è minore di b>> o <<a è uguale a b>> \\
$\geq$&$a\geq b$&maggiore o uguale&<<a è maggiore di b>> o <<a è uguale a b>>\\
\bottomrule
\end{tabular}
\caption{Disuguaglianze}
\label{tab:disuguaglianze}
\end{table}
\begin{figure}
	\centering
\begin{tikzpicture}[>=latex',line join=bevel,]
%%
\node (1) at (27bp,76.177bp) [draw,ellipse] {Una disequazione};
\node (3) at (123.25bp,18bp) [draw,ellipse] {Determinata};
\node (2) at (104.11bp,95.152bp) [draw,draw=none] {è};
\node (5) at (181.47bp,113.47bp) [draw,ellipse] {Impossibile};
\node (4) at (86bp,172.48bp) [draw,ellipse] {Sempre verificata};
\draw [->] (1) ..controls (57.307bp,83.635bp) and (62.217bp,84.843bp)  .. (2);
\draw [->] (2) ..controls (135.93bp,102.69bp) and (140.95bp,103.88bp)  .. (5);
\draw [->] (2) ..controls (110.95bp,67.577bp) and (113.8bp,56.091bp)  .. (3);
\draw [->] (2) ..controls (97.637bp,122.79bp) and (94.941bp,134.3bp)  .. (4);
\end{tikzpicture}
	\caption{Disequazione e soluzioni}
	\label{fig:DidequazioniEsoluzioni}
\end{figure}
\subsection{Principi di equivalenza per le disuguaglianze}
\label{sec:PrincipiDiEquvalenzaPerLeDisuguaglianze}
Una disuguaglianza\index{Disuguaglianza} è un confronto fra due quantità. Ovviamente è vera o è falsa.\par Consideriamo l'esempio\nobs\vref{fig:DisPgradoesempio1a},partendo da una disuguaglianza vera, sommando la stessa quantità positiva a sinistra e a destra, otteniamo una disuguaglianza ancora vera.\par  Analogo discorso con l'esempio\nobs\vref{fig:DisPgradoesempio1b}. In questo caso la disuguaglianza si mantiene vera, sommando una quantità negativa.\par
Per la moltiplicazione il discorso è quasi analogo. Nell'esempio\nobs\vref{fig:esempioDisPrimoGrado3} la disuguaglianza si mantiene vera moltiplicando entrambi i lati per una quantità positiva.\par Il discorso cambia se moltiplichiamo a sinistra e a destra per una quantità negativa. Infatti, nell'esempio\nobs\vref{fig:esempioDisPrimoGrado4}, la disuguaglianza, per mantenersi vera, deve essere invertita.
\begin{figure}
	\centering
	\begin{subfigure}[b]{.4\linewidth}
		\begin{NodesList}
			\centering
			\begin{align*}
				-3<&6\AddNode\\
				-3+2<&6+2\AddNode\\[.5cm] 
				-1<&8\AddNode
			\end{align*}
			%\tikzset{LabelStyle/.style = {left=0.1cm,pos=0.5,text=red,fill=white}}
			\LinkNodes{Sommo $+2$}%    
			\LinkNodes{\begin{minipage}[h]{3cm}
					La disuguaglianza è ancora verificata
				\end{minipage}}%
			\end{NodesList}
		%\includestandalone[width=\textwidth]{DisPromoGrado/disPrimogradoEsempio1}
		\caption{Sommando quantità positive}
		\label{fig:DisPgradoesempio1a}
	\end{subfigure}%
	\centering
	\begin{subfigure}[b]{.4\linewidth}
	\begin{NodesList}
		\centering
		\begin{align*}
			6<&8\AddNode\\
			6-3<&8-3\AddNode\\[.5cm]
			3<&5\AddNode
		\end{align*}
		%\tikzset{LabelStyle/.style = {left=0.1cm,pos=0.5,text=red,fill=white}}
		\LinkNodes{sommo $-3$}%    
		\LinkNodes{\begin{minipage}[h]{3cm}
				La disuguaglianza è ancora verificata
			\end{minipage}}%
		\end{NodesList}
		\caption{Sommando quantità negative}
		\label{fig:DisPgradoesempio1b}
	\end{subfigure}%
	\captionsetup{format=esempio,list=no}
	\caption{Diseguaglianze equivalenti per la somma}
	\label{fig:DisPgradoesempio1}
\end{figure}
\begin{figure}
	\centering
	\begin{subfigure}[b]{.4\linewidth}
	\begin{NodesList}
		\centering
		\begin{align*}
			3<&6\AddNode\\
			6\cdot 2<&6\cdot 2\AddNode\\[.5cm] 
			6<&12\AddNode
		\end{align*}
		%\tikzset{LabelStyle/.style = {left=0.1cm,pos=0.5,text=red,fill=white}}
		\LinkNodes{Moltiplico per $+2$}%    
		\LinkNodes{\begin{minipage}[h]{3cm}
				La disuguaglianza è ancora verificata
			\end{minipage}}%
		\end{NodesList}
			%\includestandalone[width=\textwidth]{DisPromoGrado/disPrimogradoEsempio1}
			\caption{Moltiplicando quantità positive}
			\label{fig:esempioDisPrimoGrado3}
		\end{subfigure}%
		\centering
		\begin{subfigure}[b]{.4\linewidth}
			\begin{NodesList}
				\centering
				\begin{align*}
					-2<&5\AddNode\\
					6\cdot(-2) <&6\cdot (-2)\AddNode\\[.5cm]
					4>&-10\AddNode
				\end{align*}
				%\tikzset{LabelStyle/.style = {left=0.1cm,pos=0.5,text=red,fill=white}}
				\LinkNodes{Moltiplico per $-2$}%    
				\LinkNodes{\begin{minipage}[h]{3cm}
						La disuguaglianza è ancora verificata
					\end{minipage}}%
				\end{NodesList}
				\caption{Moltiplicando quantità negative}
				\label{fig:esempioDisPrimoGrado4}
			\end{subfigure}%
			\captionsetup{format=esempio,list=no}
			\caption{Diseguaglianze equivalenti per il prodotto}
			\label{fig:DisuguaglianzePrimogrado2}
		\end{figure}
Per le disuguaglianze valgono tre principi elencati in seguito:
\begin{enumerate}
	\item Sommando e sottraendo la stessa espressione a entrambi i lati della disuguaglianza, ottengo una disuguaglianza equivalente.
	\item Moltiplicando e dividendo per un numero positivo diverso da zero entrambi i lati della disuguaglianza, ottengo una disuguaglianza equivalente.
	\item  Moltiplicando e dividendo per un numero negativo diverso da zero entrambi i lati della disuguaglianza, ottengo una disuguaglianza equivalente se inverto il verso della disuguaglianza.
\end{enumerate}
\section{Disequazioni di primo grado}
\label{sec:Disequuazionidiprimogrado}
\begin{definizionet}{}{}
Una disequazione\index{Disequazione} è una diseguaglianza\index{Disuguaglianza} in cui compare un'incognita.
\end{definizionet}
\begin{definizionet}{Forma normale}{}
	Una disequazione di primo grado è in forma normale\index{Disequazione!forma normale} se è scritta in una di queste forme
\begin{equation}
ax\left\{ \begin{aligned}
<b\\
\leq b\\
\geq b\\
>b
\end{aligned}\right .   
\end{equation}
\end{definizionet}
La disequazione è una disuguaglianza che è vera o falsa a seconda dei valori che sostituiamo all'incognita.
Il segno o verso di diseguaglianza divide la disequazione in due parti:il membro sinistro e quello destro.\par Una disequazione può essere o intera\index{Disequazione!intera} o frazionaria\index{Disequazione!frazionaria}, è intera se l'incognita non si trova mai al denominatore, è frazionaria se compare anche al denominatore.
\[\centering
\begin{array}{cc}
\toprule
\mathbf{Intere}  & 3x+5<2x+4  \\ [.25cm] 
  &\dfrac{3}{4}x<\dfrac{5}{2}+x+1  \\ [.25cm]
\mathbf{Frazionarie}  &\dfrac{3x+1}{2x+1}>0  \\ [.25cm]
 &\dfrac{3x+1}{x}>\dfrac{1}{2}+\dfrac{1}{2x+1}  \\ [.25cm]
\bottomrule
\end{array} 
\]
\subsection{Risolvere una disequazione di primo grado}
Per risolvere una disequazione bisogna avere chiaro cosa si intende per soluzione\index{Disequazione!soluzione}
\begin{definizionet}{Soluzione}{}
Una soluzione\index{Disequazione!soluzione} per una disequazione è un valore che sostituito all'incognita rende vera la disuguaglianza
\end{definizionet}

La definizione sembra simile a quella per le equazioni. Per un'equazione abbiamo: <<una soluzione è quel valore che rende vera l'uguaglianza>>\par
La somiglianza è solo apparente, infatti per un'equazione di primo grado in un incognita, la soluzione è un valore, per una disequazione la soluzione è un intervallo. Per esempio la disequazione elementare$X>1$ ha per soluzione tutti i numeri che sono maggiori di uno cioè l'intervallo $]1 +\infty [$.\par
Il metodo per risolvere una disequazione di primo è simile a quello per risolvere una equazione di pari grado, cioè la separazione delle variabili\index{Separazione!variabili}.\par Un esempio è il seguente. Supponiamo di dover risolvere 
\begin{esempiot}{Disequazione di primo grado}{}
\begin{equation}
3x+5<2x+6\label{equ:PrimoGradoDisequazione1}
\end{equation}
\end{esempiot}
 procediamo come nella figura\nobs\vref{fig:esempioDisequazioniPgrado1}
\begin{figure}
	\begin{NodesList}
		\centering
		\begin{align*}
			3x+5<&2x+6\AddNode\\[.5cm] 
			3x+5-2x<&6\AddNode\\[.5cm] %\AddNode[2]\\ 
			3x-2x<&6-5\AddNode\\
			x<&1\AddNode
		\end{align*}
		\LinkNodes[margin=6cm]{\begin{minipage}[h]{5cm}
				Sposto $2x$ a sinistra e cambio di segno
			\end{minipage}}
			%\LinkNodes{Sposto $2x$ a sinistra e cambio di segno}%
			\LinkNodes[margin=6cm]{\begin{minipage}[h]{5cm}
					Sposto $+5$ a destra e cambio di segno
				\end{minipage}}%
				\LinkNodes[margin=6cm]{\begin{minipage}[h]{5cm}
						Sommo
					\end{minipage}}%
				\end{NodesList}
		\captionsetup{format=esempio,list=no}
	\caption{Risoluzione disequazione\nobs\vref{equ:PrimoGradoDisequazione1}}
	\label{fig:esempioDisequazioniPgrado1}
\end{figure}

Il procedimento è  quello della risoluzione di un'equazione di primo grado, si trasportano a sinistra i valori con l'incognita, a destra i numeri, vale la stessa regola che si usa per le equazioni: spostando i termini rispettto al verso, si cambia di segno. Per rappresentare la soluzione si usa un metodo grafico che rappresenta le soluzioni. Il grafico dell'esempio è la figura\nobs\vref{fig:esempioDisequazioniPgradografico1}. Per disegnare il grafico della soluzione si procede in questa maniera: 
\begin{procedurat}{}{}
\begin{enumerate}
	\item si traccia una linea orizzontale orientata, l'asse dei numeri.
	\item si mette sotto di essa la soluzione trovata.
	\item in corrispondenza della soluzione si traccia un segmento verticale.
	\item  si guarda la soluzione e dalla parte superiore del segmento si traccia una semiretta continua nella direzione della freccia e una semiretta tratteggiata dal lato opposto.
\end{enumerate}
\end{procedurat}
\begin{figure}{I}{0pt}
	\centering
	\begin{tikzpicture}
	\draw[ -triangle 90](0,0)--(5,0);
	\draw(2,0)--(2,1);
	%%%%%soluzioni
	%%%sinistra	
	\draw[dashed](2,1)--(5,1);
	%%destra
	\draw(2,1)--(0,1);
	\node at (2,-0.5) {1};
	\end{tikzpicture}
	\captionsetup{format=grafico,list=no}
	\caption[]{Disequazione\nobs\vref{equ:PrimoGradoDisequazione1}}
	\label{fig:esempioDisequazioniPgradografico1}
\end{figure}\par Un caso leggermente più complesso è l'esempio
\begin{esempiot}{Disequazione di primo grado}{}
\begin{equation}
 3x+2\geq\dfrac{1}{2}x+3\label{equ:PrimoGradoDisequazione2}
\end{equation}
\end{esempiot}
  che vene risolto nella figura\nobs\vref{fig:esempioDisequazioniPgrado2} qui vi è un termine frazionario che può essere facilmente tolto moltiplicando entrambi i lati della disuguaglianza per il denominatore della frazione. Fatto ciò, si procede separando le incognite e sommando i termini. Al termine basta solo dividere per il numero davanti l'incognita e ottenere così il risultato. L'importante è notare che avendo moltiplicato e diviso per termini positivi, il verso della disequazione non cambia. Il grafico della disequazione è quello della figura\nobs\vref{fig:esempioDisequazioniPgradografico2} In questo caso, dato che la soluzione prevede un maggiore o uguale nel grafico è inserito un pallino $\bullet$ per indicare che il valore $\dfrac{2}{3}$ è compreso fra le soluzioni.\par
Supponiamo di dover risolvere 
\begin{esempiot}{Disequazione di primo grado}{}
\begin{equation}
3x+2>4x+3\label{equ:PrimoGradoDisequazione3}
\end{equation}
\end{esempiot}
 l'esempio\nobs\vref{fig:esempioDisequazioniPgrado3} è minimo, tuttavia nell'ultimo passaggio è importante ricordarsi che cambiando di segno si cambia di verso della diseguaglianza  avendo in questo caso moltiplicato per un termine negativo.\par
Il grafico della soluzione è il grafico\nobs\vref{fig:esempioDisequazioniPgrado3}.  
\begin{figure}
\begin{NodesList}
\centering
\begin{align*}
	3x+2\geq&\dfrac{1}{2}x+3\AddNode\\
	2(3x+2)\geq&2(\dfrac{1}{2}x+3)\AddNode\\ %[.5cm] %\AddNode[2]\\ 
	6x+4\geq&x+6\AddNode\\
	6x-x\geq&6-4\AddNode\\
	5x\geq&2\AddNode\\
	x\geq&\dfrac{2}{5}\AddNode
\end{align*}
\LinkNodes[margin=6cm]{\begin{minipage}[h]{5cm}
Moltiplico per $2x$ ed elimino la frazione
\end{minipage}}
\LinkNodes[margin=6cm]{\begin{minipage}[h]{5cm}
Semplifico
\end{minipage}}%
\LinkNodes[margin=6cm]{\begin{minipage}[h]{5cm}
Sposto $x$ e $4$ cambiando di segno
\end{minipage}}%
\LinkNodes[margin=6cm]{\begin{minipage}[h]{5cm}
Sommo
\end{minipage}}%
\LinkNodes[margin=6cm]{\begin{minipage}[h]{5cm}
Divido
\end{minipage}}%
\end{NodesList}
\captionsetup{format=esempio,list=no}\caption{Risoluzione disequazione\nobs\vref{equ:PrimoGradoDisequazione2}}
\label{fig:esempioDisequazioniPgrado2}
\end{figure}
\begin{figure}
	\centering
	\begin{tikzpicture}
	\draw[ -triangle 90](0,0)--(5,0);
	\draw(2,0)--(2,1);
	%%%%%soluzioni
	%%%sinistra	
	\draw(2,1)--(5,1);
	%%destra
	\draw[dashed](2,1)--(0,1);
	%%pallino
	\node at (2,1) {$\bullet$};
	\node at (2,-0.5) {$\dfrac{2}{5}$};
	\end{tikzpicture}
	\captionsetup{format=grafico,list=no}
	\caption{Disequazione\nobs\vref{equ:PrimoGradoDisequazione2}}
	\label{fig:esempioDisequazioniPgradografico2}
\end{figure}
\begin{figure}
	\begin{NodesList}
		\centering
		\begin{align*}
			3x+2>&4x+3\AddNode\\
			3x-4x>&3-2\AddNode\\
			-x>&1\AddNode\\
			x<&-1\AddNode
		\end{align*}
		%\LinkNodes{Sposto $2x$ a sinistra e cambio di segno}%
		\LinkNodes[margin=6cm]{\begin{minipage}[h]{5cm}
				Sposto $4x$ e $+3$ cambiando di segno
			\end{minipage}}%
			\LinkNodes[margin=6cm]{\begin{minipage}[h]{5cm}
					Sommo
				\end{minipage}}%
				\LinkNodes[margin=6cm]{\begin{minipage}[h]{5cm}
						Cambio di segno e di verso
					\end{minipage}}%
				\end{NodesList}
	\captionsetup{format=esempio,list=no}
	\caption{Risoluzione disequazione\nobs\vref{equ:PrimoGradoDisequazione3}}
	\label{fig:esempioDisequazioniPgrado3}
\end{figure}
\begin{figure}
	\centering
	\begin{tikzpicture}
	\draw[ -triangle 90](0,0)--(5,0);
	\draw(2,0)--(2,1);
	%%%%%soluzioni
	%%%sinistra	
	\draw[dashed](2,1)--(5,1);
	%%destra
	\draw(2,1)--(0,1);
	\node at (2,-0.5) {-1};
	\end{tikzpicture}
	\captionsetup{format=grafico,list=no}
	\caption{Disequazione\nobs\vref{equ:PrimoGradoDisequazione3}}
	\label{fig:esempioDisequazioniPgradografico3}
\end{figure}
\subsection{Classificare le soluzioni}
La figura\nobs\vref{fig:DidequazioniEsoluzioni} riassume la classificazione delle soluzioni per una disequazione. Possiamo avere tre casi se dopo le semplificazioni otteniamo:
\begin{enumerate}
	\item se otteniamo un risultato  del tipo $x<3$ diremo che la soluzione ottenuta è determinata\index{Soluzione!determianta}.
	\item se otteniamo un risultato del tipo $2<3$ diremo che la soluzione ottenuta è sempre verificata\index{Soluzione!indeterminata}. \'E sempre vera, e non dipende dall'incognita. 
	\item se otteniamo un risultato del tipo $5<2$ diremo che la soluzione ottenuta è impossibile\index{Soluzione!impossibile}. \'E sempre falsa, e non dipende dall'incognita.
\end{enumerate}
\section{Disequazioni frazionarie o prodotti di primo grado}
\label{DisequazioniFrazionarieProdottiPrimoGrado}
\subsection{Prodotti}
Cominciamo a introdurre il problema con un esempio
\begin{esempiot}{Disequazioni di primo grado prodotti}{}
	\begin{equation}
(x-5)(2-3x)<0\label{equ:ProdDis1}
\end{equation}
\end{esempiot}
La disequazione~\vref{equ:ProdDis1} chiede quando il prodotto\index{Disequazione!prodotto} di due binomi è negativo.  Per ottenere il segno di un prodotto bisogna conoscere il segno dei fattori\index{Fattori!segno} e li applicare la regola dei segni.\par Qui bisogna aprire una premessa. Come si è detto quando si risolve una disequazione di primo grado è possibile associare alla disequazione un grafico che esprime quando la disequazione è vera. Per esempio, banalmente, la disequazione
\begin{equation}
x-2\leq 0\label{equ:esempioDisequazioniPgradografico4}
\end{equation} ha soluzione $x\leq 2$ a cui corrisponde il grafico\nobs\vref{fig:esempioDisequazioniPgradografico4}
\begin{figure}
	\centering
	\begin{tikzpicture}
	\draw[ -triangle 90](0,0)--(5,0);
	\draw(2,0)--(2,1);
	\draw[dashed](2,1)--(5,1);
	\draw(2,1)--(0,1);
	\node at (2,1) {$\bullet$};
	\node at (2,-0.5) {2};
	\end{tikzpicture}
		\captionsetup{format=grafico,list=no}
	\caption{Disequazione\nobs\vref{equ:esempioDisequazioniPgradografico4}}
	\label{fig:esempioDisequazioniPgradografico4}
\end{figure}\par Leggendo il grafico vediamo che  per valori minori di $x$  minori di $2$ la disequazione è vera. Possiamo che $x-2$ è negativo per valori minori di due dell'incognita, positivo per valori maggiori di due e che vale zero per $x=2$. Se la disequazione è 
\begin{equation}
x-2\geq 0\label{equ:esempioDisequazioniPgradografico5}
\end{equation}
otteniamo il grafico\nobs\vref{fig:esempioDisequazioniPgradografico5}
\begin{figure}
	\centering
		\begin{tikzpicture}
		\draw[ -triangle 90](0,0)--(5,0);
		\draw(2,0)--(2,1);
		\draw(2,1)--(5,1);
		\draw[dashed](2,1)--(0,1);
		\node at (2,1) {$\bullet$};
		\node at (2,-0.5) {2};
		\end{tikzpicture}
	\captionsetup{format=grafico,list=no}
	\caption{Disequazione\nobs\vref{equ:esempioDisequazioniPgradografico5}}
	\label{fig:esempioDisequazioniPgradografico5}
\end{figure}

Il grafico ottenuto è l'opposto del precedente. La linea continua ci dice quando è vero che $x-2$ è positivo. Riflettendoci un po, questo grafico ci dice che $x-2$ è negativo per $x$ minore di due, positivo per valori maggiori di due e che vale zero per $x=2$.\par Il secondo grafico quindi, letto in maniera opportuna, ci da la soluzione anche per la precedente disequazione. Ora, per convenzione, si considera quindi che alla linea continua corrispondano valori positivi, mentre alla linea tratteggiata  valori negativi. Per evitare ambiguità  si usa per costruire il grafico che tutte le disequazioni siano del secondo tipo cioè maggiori o maggior uguale a zero.\par Costruito questo lo si legge secondo le disuguaglianze di partenza. Praticamente se devo risolvere $x-2\leq 0$ procedo in questo modo
\begin{enumerate}
	\item Risolvo  $x-2\geq 0$
	\item Costruisco il grafico\nobs\vref{fig:esempioDisequazioniPgradografico5}
	\item Dato che la disequazione generale chiede quando deve essere minore o uguale a zero, scrivo la soluzione $x\leq 0$
\end{enumerate}
  
Ritorniamo alla disequazione~\vref{equ:ProdDis1}. La disequazione è un prodotto e voglio sapere quando  è negativo. Per conoscere il segno di un prodotto bisogna conoscere il segno dei fattori che lo compongono.  Per comodità e quanto detto prima, sostituisco la disequazione con la disequazione~\vref{equ:ProdDis2} che spezzo nelle due disequazioni~\vref{equ:ProdDis2a} e~\vref{equ:ProdDis2b} 
% \begin{subequations}
% 	\begin{align}
% 	(x-5)(2-3x)>0\label{equ:ProdDis2}
% 	\intertext{formata dalla disequazione}	
% 	(x-5)>0\label{equ:ProdDis2a}
% 	\intertext{e dalla disequazione}
% 	(2-3x)>0\label{equ:ProdDis2b}
% 	\end{align}
% \end{subequations}
\begin{subequations}
	\begin{equation}
	(x-5)(2-3x)>0\label{equ:ProdDis2}
	\end{equation}
	formata dalla disequazione
	\begin{equation}
	(x-5)>0\label{equ:ProdDis2a}
	\end{equation}
	e dalla disequazione
	\begin{equation}
	(2-3x)>0\label{equ:ProdDis2b}
	\end{equation}
\end{subequations}
Risolvo la disequazione\nobs\vref{equ:ProdDis2a}. La disequazione ha per soluzione $x>5$ e per grafico\nobs\vref{fig:ProdottoDis2a}
\begin{figure}
	\centering
	\begin{subfigure}[b]{.4\linewidth}
		\begin{tikzpicture}
		\draw[ -triangle 90](0,0)--(5,0);
		\draw(2,0)--(2,1);
		\draw(2,1)--(5,1);
		\draw[dashed](2,1)--(0,1);
		%\node at (2,1) {$\bullet$};
		\node at (2,-0.5) {$5\vphantom{\dfrac{2}{3}}$};
		%\node at (2,-0.5) {5};
		\end{tikzpicture}
%		\renewcommand\thesubfigure{Grafico \thefigure\alph{subfigure}}
		\caption{Disequazione\nobs\vref{equ:ProdDis2a}}
		\label{fig:ProdottoDis2a}
	\end{subfigure}%
	\centering
	\begin{subfigure}[b]{.4\linewidth}
		\centering
		\begin{tikzpicture}
		\draw[ -triangle 90](0,0)--(5,0);
		\draw(2,0)--(2,1);
		\draw[dashed](2,1)--(5,1);
		%\node at (2,1) {$\bullet$};
		\draw(2,1)--(0,1);
		\node at (2,-0.5) {$\dfrac{2}{3}$};
		\end{tikzpicture}
	%	\renewcommand\thesubfigure{Grafico \thefigure\alph{subfigure}}
		\caption{Disequazione\nobs\vref{equ:ProdDis2b}}
		\label{fig:ProdottoDis2b}
	\end{subfigure}%
		\qquad\qquad\centering
		\begin{subfigure}[b]{.4\linewidth}
			\centering
				\begin{tikzpicture}
				\draw[ -triangle 90](0,0)--(5,0);
				\draw(2,0)--(2,1);
				\draw[dashed](2,1)--(5,1);
				%\node at (2,1) {$\bullet$};
				\draw(2,1)--(0,1);
				\node at (2,-0.5) {$\dfrac{2}{3}$};
				\draw(3,0)--(3,2);
				\draw(3,2)--(5,2);
				\draw[dashed](3,2)--(0,2);
				\node at (3,-0.5) {$5$};
				%\node at (3,2) {$\bullet$};
				\end{tikzpicture}
		%	\renewcommand\thesubfigure{Grafico \thefigure\alph{subfigure}}
			\caption{Disequazione\nobs\vref{equ:ProdDis2}}
			\label{fig:ProdottoDis2c}
		\end{subfigure}%
		\captionsetup{format=grafico,list=no}
	\caption{Disequazione\nobs\vref{equ:ProdDis2}}
\end{figure}

Mentre la  disequazione\nobs\vref{equ:ProdDis2a} ha per soluzione $x<\dfrac{2}{3}$ con grafico\nobs\vref{fig:ProdottoDis2b}
 
Interessante è il grafico\nobs\vref{fig:ProdottoDis2c} che riunisce i due precedenti. Nel grafico, l'asse delle $x$ è diviso in tre parti, prima di $\dfrac{2}{3}$, fra $\dfrac{2}{3}$ e $5$ e dopo il $5$. Prima di $\dfrac{2}{3}$, guardando il grafico,  è positiva la disequazione\nobs\vref{equ:ProdDis2b} (linea continua)  ed è negativa la disequazione\nobs\vref{equ:ProdDis2a} (linea tratteggiata). Quindi il loro prodotto è negativo. Per valori compresi fra $\dfrac{2}{3}$ e $5$ entrambe le disequazioni  sono negative, quindi il loro prodotto è positivo. Dopo il $5$ è positiva\nobs\vref{equ:ProdDis2a} ed è negativa\nobs\vref{equ:ProdDis2b}. 

Per rispondere finalmente, alla disequazione\nobs\vref{equ:ProdDis1} basta leggere il grafico precedente e vedere che il segno del prodotto è negativo per valori di $x<\dfrac{2}{3}$ e per valori di $x>5$

Un altro esempio risolviamo passo passo la disequazione
\begin{esempiot}{Disequazione prodotto}{}
\begin{equation}
(2-x)(3x-1)(\dfrac{2}{3}-x)\leq 0\label{equ:ProdDis3}
\end{equation}
\end{esempiot}
suddivido la disequazione in tre parti che verranno risolte a parte.
%
\begin{subequations}
	\begin{equation}
	(2-x)\geq 0\label{equ:ProdDis3a}
	\end{equation}
	\begin{equation}
	(3x-1)\geq 0\label{equ:ProdDis3b}
	\end{equation}
	\begin{equation}
	(\dfrac{2}{3}-x)\geq 0\label{equ:ProdDis3c}
	\end{equation}
\end{subequations}
Iniziamo con il risolvere  la disequazione\nobs\vref{equ:ProdDis3a}. Utilizzando  il procedimento\nobs\vref{svo:ProDis3a} e otteniamo il grafico\nobs\vref{graf:ProDis3a}. Continuiamo con la disequazione\nobs\vref{equ:ProdDis3b} dal  procedimento\nobs\vref{svo:ProDis3b} si ha il grafico\nobs\vref{graf:ProDis3b}. Terminiamo  con la disequazione\nobs\vref{equ:ProdDis3c} dal  procedimento\nobs\vref{svo:ProDis3c} si ottiene il grafico\nobs\vref{graf:ProDis3c}. Non resta che riunire i tre grafici nel grafico\nobs\vref{equ:ProdDis3c}. L'asse delle $x$ è suddiviso in quattro parti. Prima di $\dfrac{1}{3}$, tra $\dfrac{1}{3}$ e $\dfrac{2}{3}$, tra $\dfrac{2}{3}$ e $2$ ed infine dopo $2$. Prima di $\dfrac{1}{3}$ abbiamo due linee continue ed una tratteggiata quindi $(+)\cdot(+)\cdot(-)=-$. Tra $\dfrac{1}{3}$ e $\dfrac{2}{3}$ abbiamo tre linee continue $(+)\cdot(+)\cdot(+)=+$. Tra $\dfrac{2}{3}$ e $2$ abbiamo una linea continua, una tratteggiata e una linea continua. Dopo $2$ abbiamo due linee tratteggiate ed una continua $(-)\cdot(-)\cdot(+)=+$.  La disequazione\nobs\vref{equ:ProdDis3} chiede quando il prodotto è negativo, riguardando quello che si è detto, la risposta è $x\leq \dfrac{1}{3}$ e $\dfrac{2}{3}\leq x \leq 2$.
\begin{figure}
	\centering
	\begin{subfigure}[]{\linewidth}
		\begin{NodesList}
			\begin{align*}
				2-x\geq 0&\AddNode\\%
				-x\geq-2&\AddNode\\%
				x\leq 2&\AddNode%
			\end{align*}
			\LinkNodes[margin=6cm]{}%
			\LinkNodes[margin=6cm]{}%
		\end{NodesList}
	%	\renewcommand\thesubfigure{Svolgimento \thefigure\alph{subfigure}}
		\caption{Risoluzione disequazione}
		\label{svo:ProDis3a}
	\end{subfigure}%
	\qquad
	\begin{subfigure}[]{\linewidth}
		\centering
		\begin{tikzpicture}
		\draw[ -triangle 90](0,0)--(5,0);
		\draw(2,0)--(2,1);
		\draw[dashed](2,1)--(5,1);
		\node at (2,1) {$\bullet$};
		\draw(2,1)--(0,1);
		\node at (2,-0.5) {$2$};
		\end{tikzpicture}
	%	\renewcommand\thesubfigure{Grafico \thefigure\alph{subfigure}}
		\caption{Grafico disequazione}
		\label{graf:ProDis3a}
	\end{subfigure}%
	\captionsetup{format=esempio,list=no}
	\caption{Disequazione\nobs\vref{equ:ProdDis3a} }
	\label{esempio:ProDisa3a}
\end{figure} 
\begin{figure}
	\centering
	\begin{subfigure}[]{\linewidth}
		\begin{NodesList}
			\centering
			\begin{align*}
				3x-1\geq& 0\AddNode\\
				3x\geq&1\AddNode\\
				x\geq&\dfrac{1}{3}\AddNode
			\end{align*}
			%\LinkNodes{Sposto $2x$ a sinistra e cambio di segno}%
			\LinkNodes[margin=6cm]{}%
			\LinkNodes[margin=6cm]{}%
		\end{NodesList}
		\caption{Risoluzione disequazione}
		\label{svo:ProDis3b}
	\end{subfigure}%
	\qquad
	\begin{subfigure}[]{\linewidth}
		\centering
		\begin{tikzpicture}
		\draw[ -triangle 90](0,0)--(5,0);
		\draw(2,0)--(2,1);
		\draw(2,1)--(5,1);
		\node at (2,1) {$\bullet$};
		\draw[dashed](2,1)--(0,1);
		\node at (2,-0.5) {$\dfrac{1}{3}$};
		\end{tikzpicture}
		\caption{Grafico disequazione}
		\label{graf:ProDis3b}
	\end{subfigure}%
	\captionsetup{format=esempio,list=no}
	\caption{Disequazione\nobs\vref{equ:ProdDis3b}}
	\label{esempio:ProDisa3b}
	\end{figure}
\begin{figure}
	\centering
	\begin{subfigure}[]{\linewidth}
		\begin{NodesList}
			\centering
			\begin{align*}
				\dfrac{2}{3}-x\geq& 0\AddNode\\
				-x\geq&-\dfrac{2}{3}\AddNode\\
				x\leq& \dfrac{2}{3}\AddNode
			\end{align*}
			%\LinkNodes{Sposto $2x$ a sinistra e cambio di segno}%
			\LinkNodes[margin=6cm]{}%
			\LinkNodes[margin=6cm]{}%
		\end{NodesList}
		\caption{Risoluzione disequazione}
		\label{svo:ProDis3c}
	\end{subfigure}%
	\qquad
	\begin{subfigure}[]{\linewidth}
		\centering
		\begin{tikzpicture}
		\draw[ -triangle 90](0,0)--(5,0);
		\draw(2,0)--(2,1);
		\draw[dashed](2,1)--(5,1);
		\node at (2,1) {$\bullet$};
		\draw(2,1)--(0,1);
		\node at (2,-0.5) {$\dfrac{2}{3}$};
		\end{tikzpicture}
		\caption{Grafico disequazione}
		\label{graf:ProDis3c}
	\end{subfigure}%
	\captionsetup{format=esempio,list=no}
	\caption{Disequazione\nobs\vref{equ:ProdDis3c}}
	\label{esempio:ProDisa3c}
\end{figure}
\begin{figure}
	\centering
		\begin{tikzpicture}
		\draw[ -triangle 90](0,0)--(6,0);
		\draw(2,0)--(2,1);
		\draw(2,1)--(6,1);
		\node at (2,1) {$\bullet$};
		\draw[dashed](2,1)--(0,1);
		\node at (2,-0.5) {$\dfrac{1}{3}$};
		\draw(3,0)--(3,2);
		\draw[dashed](3,2)--(6,2);
		\draw(3,2)--(0,2);
		\node at (3,-0.5) {$\dfrac{2}{3}$};
		\node at (3,2) {$\bullet$};
		\draw(4,0)--(4,3);
		\draw[dashed](4,3)--(6,3);
		\draw(4,3)--(0,3);
		\node at (4,-0.5) {$2$};
		\node at (4,3) {$\bullet$};
		\end{tikzpicture}
	\captionsetup{format=grafico,list=no}
	\caption{Disequazione\nobs\vref{equ:ProdDis3}}
	\label{graf:ProDis3d}
\end{figure}
\subsection{Frazioni}
Iniziamo con il definire una disequazione frazionaria di primo grado in forma normale.
\begin{definizionet}{Disequazione frazionaria di primo grado}{}
Una disequazione frazionaria\index{Disequazione!frazionaria} è una disequazione del tipo 
\begin{equation}
\dfrac{ax+b}{cx+d}\left\{ \begin{aligned}
<0\\
\leq 0\\
\geq 0\\
>0
\end{aligned}\right .   
\end{equation}
\end{definizionet}
Supponiamo di dover risolvere
\begin{esempiot}{Disequazione fratta}{}
 \begin{equation}
\dfrac{3x+1}{1-x}\leq 0\label{equ:DisFrazPrimoG1}
\end{equation}
\end{esempiot}
Una disequazione frazionaria si risolve come le precedenti disequazioni. Viene anche qui usata la regola dei segni. Si parte dal segno del denominatore e si confronta con il segno del numeratore.\par Solo un appunto, prima di procedere con la risoluzione della disequazione bisogna ricordarsi che una disequazione frazionaria è una frazione  e una frazione esiste se il suo denominatore è diverso da zero. In questo caso la frazione esiste se $1-x$ non vale zero. Qui è evidente che $1-x$ è zero se $x=1$.\par Il procedimento è quello solito,suddivido la frazione nelle sue parti e risolvo due disequazioni, anche qui cercando valori positivi. Avremo
\begin{subequations}
	\begin{equation}
	3x+1\geq 0\label{equ:DisFrazPrimoG1a} 
	\end{equation}
\begin{equation}
1-x> 0\label{equ:DisFrazPrimoG1b} 
\end{equation}
\end{subequations}   
Iniziamo con il risolvere la disequazione\nobs\vref{equ:DisFrazPrimoG1a}. Dallo svolgimento\nobs\vref{svo:DisFrazPrimoG1a} otteniamo il grafico\nobs\vref{graf:DisFrazPrimoG1a}. 
\begin{figure}
	\centering
	\begin{subfigure}[]{\linewidth}
		\begin{NodesList}
			\centering
			\begin{align*}
				3x+1\geq& 0\AddNode\\
				3x\geq&-1\AddNode\\
				x\geq&-\dfrac{1}{3}\AddNode
			\end{align*}
			%\LinkNodes{Sposto $2x$ a sinistra e cambio di segno}%
			\LinkNodes[margin=6cm]{}%
			\LinkNodes[margin=6cm]{}%
		\end{NodesList}
		\caption{Risoluzione disequazione}
		\label{svo:DisFrazPrimoG1a}
	\end{subfigure}%
	\qquad
	\begin{subfigure}[]{\linewidth}
		\centering
		\begin{tikzpicture}
		\draw[ -triangle 90](0,0)--(5,0);
		\draw(2,0)--(2,1);
		\draw(2,1)--(5,1);
		\node at (2,1) {$\bullet$};
		\draw[dashed](2,1)--(0,1);
		\node at (2,-0.5) {$-\dfrac{1}{3}$};
		\end{tikzpicture}
		\caption{Grafico disequazione}
		\label{graf:DisFrazPrimoG1a}
	\end{subfigure}%
	\captionsetup{format=esempio,list=no}
	\caption{Disequazione\nobs\vref{equ:DisFrazPrimoG1a}}
	\label{esempio:DisFrazPrimoG1a}
\end{figure}
     \begin{figure}
	\centering
	\begin{tikzpicture}
	\draw[ -triangle 90](0,0)--(5,0);
	\draw(2,0)--(2,1);
	\draw(2,1)--(5,1);
	\node at (2,1) {$\bullet$};
	\draw[dashed](2,1)--(0,1);
	\node at (2,-0.5) {$-\dfrac{1}{3}$};
	\draw(3,0)--(3,2);
	\draw[dashed] ( 3,2)--(5,2);
	\draw(3,2)--(0,2);
	\node at (3,-0.5) {$1$};
	%\node at (3,2) {$\bullet$};
	\end{tikzpicture}
	\captionsetup{format=esempio,list=no}
	\caption{Disequazione\nobs\vref{equ:DisFrazPrimoG1}}
	\label{graf:DisFrazPrimoG1}
\end{figure}
Continuiamo con la disequazione\nobs\vref{equ:DisFrazPrimoG1b}.
Questa disequazione è differente dalla precedente perché si passa da un maggiore o uguale a zero ad un maggiore di zero. Infatti tale termine corrisponde al denominatore della frazione e un denominatore non può essere mai uguale a zero. Anche 
qui dallo svolgimento\nobs\vref{svo:DisFrazPrimoGìb} otteniamo il grafico\nobs\vref{graf:DisFrazPrimoG1b}. Unendo i due grafici otteniamo il grafico\nobs\vref{graf:DisFrazPrimoG1}. Abbiamo tre zone prima di $-\dfrac{1}{3}$, fra $-\dfrac{1}{3}$ e $1$  e dopo $1$. Prima di $-\dfrac{1}{3}$ abbiamo una linea continua ed una linea tratteggiata quindi $(+)\cdot(-)=-$. Fra $-\dfrac{1}{3}$ e $1$ abbiamo due linee continue quindi $(+)\cdot(+)=+$. Dopo $1$ abbiamo una linea tratteggiata ed una continua quindi $(-)\cdot(+)=-$. La disequazione\nobs\vref{equ:DisFrazPrimoG1} chiede quando il rapporto è negativo e guardando i precedenti risultati la risposta è $x\leq -\dfrac{1}{3}$ e $x>1$, maggiore e non uguale perché riferita al denominatore.  

\begin{figure}
	\centering
	\begin{subfigure}[]{\linewidth}
		\begin{NodesList}
			\centering
			\begin{align*}
				1-x>& 0\AddNode\\
				-x>&-1\AddNode\\
				x<&1\AddNode
			\end{align*}
			%\LinkNodes{Sposto $2x$ a sinistra e cambio di segno}%
			\LinkNodes[margin=6cm]{}%
			\LinkNodes[margin=6cm]{}%
		\end{NodesList}
		\caption{Risoluzione disequazione}
		\label{svo:DisFrazPrimoGìb}
	\end{subfigure}%
	\qquad
	\begin{subfigure}[]{\linewidth}
		\centering
		\begin{tikzpicture}
		\draw[ -triangle 90](0,0)--(5,0);
		\draw(2,0)--(2,1);
		\draw[dashed](2,1)--(5,1);
		\node at (2,1) {$\bullet$};
		\draw(2,1)--(0,1);
		\node at (2,-0.5) {$1$};
		\end{tikzpicture}
		\caption{Grafico disequazione}
		\label{graf:DisFrazPrimoG1b}
	\end{subfigure}%
	\captionsetup{format=esempio,list=no}
	\caption{Disequazione\nobs\vref{equ:DisFrazPrimoG1b}}
	\label{esempio:DisFrazPrimoG1b}
\end{figure}

Un esempio più complesso è il seguente
\begin{esempiot}{Disequazione fratta}{}
\begin{equation}
\dfrac{1}{x+2}\geq-\dfrac{3}{2x+1}\label{equ:DisFrazPrimoG2}
\end{equation}
\end{esempiot}

La disequazione è diversa dalle precedenti, abbiamo due termini frazionari quindi prima di risolverla bisognerà prima discuterla poi semplificarla e quindi  risolverla. L'esempio\nobs\vref{esempio:DisFrazPrimoG2a} mostra quanto detto. 

Resta da risolvere la disequazione
\begin{equation}
\dfrac{5x+7}{(x+2)(2x+1)}\geq 0 \label{equ:DisFrazPrimoG2a}
\end{equation}
Questa disequazione è formata da tre parti 
\begin{subequations}
	\begin{equation}
	5x+7\geq 0\label{equ:DisFrazPrimoG2aP1}
	\end{equation}
	\begin{equation}
	x+2>0\label{equ:DisFrazPrimoG2aP2}
	\end{equation}
	\begin{equation}
	2x+1> 0\label{equ:DisFrazPrimoG2aP3}
	\end{equation}
\end{subequations}
Avremo i seguenti risultati che ci permettono di costruire il grafico\nobs\vref{equ:DisFrazPrimoG2} della disequazione. 
\begin{align*}
	5x+7\geq& 0 & x\geq&-\dfrac{7}{5}\\
	x+1>&0&x>&-1\\
	2x+1>&0&x>&-\dfrac{1}{2}
\end{align*}
\begin{figure}
	\centering
	\begin{tikzpicture}
	\draw[ -triangle 90](0,0)--(6,0);
	\draw(2,0)--(2,1);
	\draw(2,1)--(6,1);
	\node at (2,1) {$\bullet$};
	\draw[dashed](2,1)--(0,1);
	\node at (2,-0.5) {$-\dfrac{7}{5}$};
	\draw(3,0)--(3,2);
	\draw(3,2)--(6,2);
	\draw[dashed](3,2)--(0,2);
	\node at (3,-0.5) {$-1$};
	%\node at (3,2) {$\bullet$};
	\draw(4,0)--(4,3);
	\draw(4,3)--(6,3);
	\draw[dashed](4,3)--(0,3);
	\node at (4,-0.5) {$-\dfrac{1}{2} $};
	%\node at (4,3) {$\bullet$};
	\end{tikzpicture}
	\captionsetup{format=grafico,list=no}
	\caption[]{Disequazione\nobs\vref{equ:DisFrazPrimoG2}}
	\label{graf:DisFrazPrimoG2}
\end{figure}
\begin{figure}
		\centering
\begin{minipage}{\linewidth}
\begin{NodesList}
	\begin{align*}
		\dfrac{1}{x+2}\geq-\dfrac{3}{2x+1}&\AddNode\AddNode[2]\AddNode[3]\\
		&\\
		\left .\begin{aligned}
			 x+2= 0&\\
			 x=-2&\\
			 x\neq 2&\\
			 2x+1= 0&\\
			 2x=-1&\\
			 x=-\dfrac{1}{2}&\\
			 x\neq-\dfrac{1}{2}&& 
		\end{aligned}\right\}&\qquad\text{Discussione}\AddNode\\
		&\\
		\left .\begin{aligned}
			\dfrac{2x+1\geq -3(x+2)}{(x+2)(2x+1)}&\\
			\dfrac{2x+1\geq -3x-6}{(x+2)(2x+1)}&\\
			\dfrac{3x+2x+1+6}{(x+2)(2x+1)}\geq 0&\\
			\dfrac{5x+7}{(x+2)(2x+1)}\geq 0 &&
		\end{aligned}\right\}&\qquad \text{Semplificazione }\AddNode[2]\\
		&\\
	\left .\begin{aligned}
		\dfrac{5x+7}{(x+2)(2x+1)}\geq 0 &&
	\end{aligned}\right\}& \qquad \text{Risoluzione}\AddNode[3]
\end{align*}
	{\tikzset{ArrowStyle/.style={>=triangle 90,->}}
	\tikzset{LabelStyle/.style = {left=0.1cm,pos=.4,text=red}}
	\LinkNodes{}%
	\LinkNodes{}
\LinkNodes{}} %
\end{NodesList}
\end{minipage}
	\captionsetup{format=esempio,list=no}
	\caption{Disequazione\nobs\vref{equ:DisFrazPrimoG2}}
	\label{esempio:DisFrazPrimoG2a}
\end{figure}
Le    
disequazioni\nobs\vrefrange{equ:DisFrazPrimoG2aP2}{equ:DisFrazPrimoG2aP3} sono maggiori di zero e non maggiori e uguali a zero perché sono nel denominatore della frazione. Il grafico della disequazione è diviso in quattro parti: prima di $x<-\dfrac{7}{5} $, qui abbiamo tre linee tratteggiate quindi $(-)\cdot(-)\cdot(-)=-$, $-\dfrac{7}{5}<x<-1$ qui abbiamo due linee tratteggiate e una continua $(-)\cdot(-)\cdot(+)=+$, $x>-\dfrac{1}{2} $ qui abbiamo una linea tratteggiata  e due linee continue $(-)\cdot(+)\cdot(+)=-$, $x>-\dfrac{1}{2}$ qui abbiamo tre linee continue $(+)\cdot(+)\cdot(+)=+$. La disequazione\nobs\vref{equ:DisFrazPrimoG2a} chiede quando la frazione sia maggiore o uguale a zero quindi, per quanto detto prima le soluzioni sono $-\dfrac{7}{5}\leq x<-1$ e $x>-\dfrac{1}{2}$.


\subsection{Riepilogo}
\begin{procedurat}{}{}
\begin{enumerate}
	\item Verifico se la disequazione è in forma normale. Altrimenti semplifico l'espressione.
	\item Separo il numeratore e il denominatore della frazione e li pongo maggiori di zero.
	\item Risolvo separatamente le due disequazioni.
	\item Sovrappongo i grafici delle due disequazioni.
	\item Applico la regola dei segni.
	\item Risolvo la disequazione confrontando i risultati del grafico da quanto richiesto dalla disequazione.
\end{enumerate}
\end{procedurat}
 


% !TeX encoding = UTF-8
% !TeX spellcheck = it_IT
% !TeX root = tabellequarto.tex
\chapter{Funzioni razionali e irrazionali}
\section{Definizioni}
\begin{definizionet}{Campo di esistenza}{}
Data una funzione $\funzione{f}{\R}{\R}$ il campo di esistenza \PA{C.E.} è l'insieme dei valori che è possibile assegnare alla funzione.\index{Funzione!campo!esistenza}
\end{definizionet}
\begin{definizionet}{Positività di una funzione}{}
	Data una funzione $\funzione{f}{\R}{\R}$, la positività\index{Funzione!positività} di una funzione è l'insieme dei valore del \PA{C.E.} per cui\[f(x)\geq 0 \]
\end{definizionet}
\section{Razionale intera}
\subsection{Definizione}
\begin{definizionet}{Razionale intera}{}
Una funzione $\funzione{f}{\R}{\R}$,  è  razionale intera se per l'incognita,  compaiono solo le operazioni di somma, prodotto e potenza.\index{Funzione!razionale!intera}
\end{definizionet}
\begin{esempiot}{Razionale intera}{}
	\begin{align*}
	f(x)=&2x^2+3x+4&f(x)=&2\sqrt{2}x^3+x+1\\
		f(x)=&8x^3+4x+2&f(x)=&x^2+2x+\sqrt{6}\\
		f(x)=&\frac{1}{2}x^5+2x+\frac{2}{3}&f(x)=&2x^0\\
	\end{align*}
\end{esempiot}
\subsection{Campo di esistenza}
Una funzione razionale intera per come è formata, ad  ogni valore assegnato all'incognita questa è sempre  definita. Quindi la funzione esiste per ogni numero appartenente ai reali. Per questo possiamo scrivere:
\begin{definizionet}{Campo di esistenza razionale intera}{}
Il campo di esistenza \PA{C.E.} di una funzione razionale intera $\funzione{f}{\R}{\R}$, è tutto l'insieme dei numeri reali.
\[\PA{C.E.}=\lbrace\forall\ x\in\R\rbrace\]
\end{definizionet} 

\section{Razionale fratta}
\subsection{Definizione}
\begin{definizionet}{Razionale fratta}{}
	Una funzione $\funzione{f}{\R}{\R}$ è razionale fratta se l'incognita compare al denominatore di una frazione.\index{Funzione!razionale!fratta}
	\[f(x)=\dfrac{N(x)}{D(x)}\]
\end{definizionet}
\begin{esempiot}{Razionale fratta}{}
	\begin{align*}
	f(x)=&\dfrac{1}{x+1}&f(x)=&\dfrac{\sqrt{2}}{x^2-3x}\\
	f(x)=&\dfrac{x^2+5x+8}{x^3+x^4}&f(x)=&\dfrac{\sqrt{2}x+2x^3}{x^2+x+1}
	\end{align*}
\end{esempiot}
\subsection{Campo di esistenza}
In una razionale fratta, l'incognita è al denominatore. In una frazione il denominatore non può essere uguale a zero. Detto questo l'incognita non può assumere valori che annullano il denominatore.
\begin{definizionet}{Campo di esistenza razionale fratta}{}
	Il campo di esistenza \PA{C.E.} di una funzione razionale fratta $\funzione{f}{\R}{\R}$, è  l'insieme dei numeri reali per cui il denominatore è diverso da zero.
	\[\text{\PA{C.E.}}=\R-\lbrace\forall\ x\in\R\ \text{tali che} D(x)= 0\rbrace\]
\end{definizionet}
\begin{esempiot}{Campo di esistenza razionale fratta}{}
	Trovare il campo di esistenza della funzione\[f(x)=\dfrac{x-4}{x^2+5x+4}\]
	Procediamo nel  seguente modo
	\begin{enumerate}
		\item Poniamo il denominatore uguale a zero e risolviamo l'equazione\begin{align*}
		x^2+5x+4=&0\\
		x_{1,2}=&\dfrac{-5\pm\sqrt{25-4(1)(4)}}{2}\\
		=&\dfrac{-5\pm\sqrt{9}}{2}\\
		=&\dfrac{-5\pm 3}{2}\\
		=&\begin{cases}
		x_1=\dfrac{-5-3}{2}=-4\\
		x_2=\dfrac{-5+3}{2}=-1
		\end{cases}
		\end{align*}
		\item Escludiamo i valori trovati\[\text{\PA{C.E.}}=\R-\lbrace -1,-4\rbrace\]
	\end{enumerate}
\end{esempiot}
\begin{esempiot}{Campo di esistenza razionale fratta}{}
	Trovare il campo di esistenza della funzione\[f(x)=\dfrac{x^2+1}{x^2+2x+4}\]
	Procediamo nel  seguente modo:
	\begin{enumerate}
		\item Poniamo il denominatore uguale a zero e risolviamo l'equazione\begin{align*}
		x^2+2x+4=&0\\
		x_{1,2}=&\dfrac{-2\pm\sqrt{4-4(1)(4)}}{2}\\
		=&\dfrac{-2\pm\sqrt{-12}}{2}\\
	\intertext{L'equazione non ha soluzioni}
		\end{align*}
		\item Non abbiamo valori da escludere\[\text{\PA{C.E.}}=\R\]
	\end{enumerate}
\end{esempiot}
\section{Irrazionale pari intera}
\subsection{Definizione}
\begin{definizionet}{Irrazionale pari intera}{}
	Una funzione $\funzione{f}{\R}{\R}$ è irrazionale pari intera se l'incognita appartiene al radicando di una radice di indice pari e  non si trova nel denominatore di una frazione.\index{Funzione!irrazionale!pari intera}
\end{definizionet}
\begin{esempiot}{Irrazionale pari intera}{}
\begin{align*}
f(x)=&\sqrt{x}+1&f(x)=&\sqrt{x^2+3x+1}\\
f(x)=&\sqrt[6]{x^3+3x}+x&f(x)=&\sqrt{x^2+3x+1}\\
\end{align*}
\end{esempiot}
\subsection{Campo di esistenza}
Una radice di indice pari è definita quando il radicando è positivo.  
\begin{definizionet}{Campo di esistenza irrazionale intera}{}
	Il campo di esistenza \PA{C.E.} di una funzione irrazionale intera $\funzione{f}{\R}{\R}$, è  l'insieme dei numeri reali per cui il radicando è maggiore o uguale a  zero.
\end{definizionet} 
\begin{esempiot}{Campo di esistenza irrazionale intera}{}
Determinare il campo di esistenza della funzione\[f(x)=\sqrt{x^2+3x}\]Poniamo il radicando maggiore o uguale a zero e procediamo nel seguente modo:
\begin{align*}
x^2+3x\geq& 0\\
\intertext{Risolviamo l'equazione}
x^2+3x=& 0\\
x_{1,2}=&\dfrac{-3\pm\sqrt{9-0}}{2}\\
=&\begin{cases}
x_1=\dfrac{-3-3}{2}=-3\\
\\
x_2=\dfrac{-3+3}{2}=0\\
\end{cases}
\end{align*}
Otteniamo il grafico
\begin{center}
\includestandalone[width=8.5cm]{quarto/DisSecGrado/FunzIrraEsempio1}
\end{center}
Quindi il \PA{C.E.}	è
\[x\leq -3\quad 0\leq x\]
\end{esempiot}
\section{Irrazionale pari fratta}
\subsection{Definizione}
\begin{definizionet}{Irrazionale pari fratta}{}
	Una funzione $\funzione{f}{\R}{\R}$ è irrazionale pari intera se l'incognita appartiene al radicando di una radice di indice pari e   si trova nel denominatore di una frazione. \index{Funzione!irrazionale!pari fratta}
\end{definizionet}
\begin{esempiot}{Irrazionale pari fratta}{}
	\begin{align*}
	f(x)=&\frac{1}{\sqrt{x}}+1&f(x)=&\sqrt[4]{\frac{x-2}{x^2+2}}\\
	f(x)=&\frac{\sqrt{x^2+1}}{x}&f(x)=&\frac{x}{\sqrt[6]{x^2-3}}\\
	\end{align*}
\end{esempiot}
\subsection{Campo di esistenza}
Una funzione irrazionale pari e fratta deve unire entrambi le caratteristiche. Avremo:
\begin{definizionet}{Campo di esistenza irrazionale fratta}{}
	Il campo di esistenza \PA{C.E.} di una funzione irrazionale fratta $\funzione{f}{\R}{\R}$, è  l'insieme dei numeri reali per cui
	\begin{enumerate}
		\item Il radicando deve essere maggiore o uguale a zero.
		\item Il denominatore è diverso da zero.
	\end{enumerate}
\end{definizionet} 
\begin{esempiot}{Campo di esistenza irrazionale pari fratta}{}
	Determinare il campo di esistenza della funzione\[f(x)=\sqrt{\dfrac{x^2-1}{x^2-4}}\] Procediamo nel seguente modo:
	\begin{align*}
	f(x)=&\sqrt{\dfrac{x^2-1}{x^2-4}}\\
	\intertext{Radicando maggiore o uguale a zero}
	\dfrac{x^2-1}{x^2-4}\geq&0\\	
	x^2-1\geq&0\\
	x^2-1=&0\\
	x_{1,2}=&\dfrac{0\pm\sqrt{0-4(-1)(1)}}{2}\\
=&\dfrac{\pm\sqrt{4}}{2}=\begin{cases}
x_1=+\dfrac{2}{2}=+1\\
\\
x_2=-\dfrac{2}{2}=-1\\
\end{cases}	
	\intertext{Denominatore diverso da zero}
x^2-4>&0\\
x^2-4=&0\\
x_{1,2}=&\dfrac{0\pm\sqrt{0-4(-4)(1)}}{2}\\
=&\dfrac{\pm\sqrt{16}}{2}=\begin{cases}
x_1=+\dfrac{4}{2}=+2\\
\\
x_2=-\dfrac{4}{2}=-2\\
\end{cases}	
	\end{align*}
	Otteniamo il grafico
	\begin{center}
	\includestandalone[width=8.5cm]{quarto/DisSecGrado/FunzIrraEsempio2}
	\end{center}
Quindi il \PA{C.E.}	è
\[ x< -2\quad -1\leq x\leq 1\quad 2<x  \]
\end{esempiot}
\begin{esempiot}{Campo di esistenza irrazionale pari fratta}{}
		Determinare il campo di esistenza della funzione\[f(x)=\dfrac{x^2-1}{\sqrt{x^2-4}}\]Procediamo nel seguente modo:
		\begin{align*}
		f(x)=&\dfrac{x^2-1}{\sqrt{x^2-4}}
		\intertext{Radicando maggiore o uguale a zero}
		\intertext{Denominatore diverso da zero}
	x^2-4>&0\\
	x^2-4=&0\\
	x_{1,2}=&\dfrac{0\pm\sqrt{0-4(-4)(1)}}{2}\\
	=&\dfrac{\pm\sqrt{16}}{2}=\begin{cases}
	x_1=+\dfrac{4}{2}=+2\\
	\\
	x_2=-\dfrac{4}{2}=-2\\
	\end{cases}	
		\end{align*}
			Otteniamo il grafico
		\begin{center}
			\includestandalone[width=8.5cm]{quarto/DisSecGrado/FunzIrraEsempio3}
		\end{center}
		Quindi il \PA{C.E.}	è
		\[ x< -2\quad 2<x  \]
\end{esempiot}
\begin{esempiot}{Campo di esistenza irrazionale pari fratta}{}
	Determinare il campo di esistenza della funzione\[f(x)=\dfrac{\sqrt{x^2-1}}{x^2-4}\]Procediamo nel seguente modo:
	\begin{align*}
	f(x)=&\dfrac{\sqrt{x^2-1}}{x^2-4}\\
	\intertext{Radicando maggiore o uguale a zero}	
	x^2-1\geq&0\\
	x^2-1=&0\\
	x_{1,2}=&\dfrac{0\pm\sqrt{0-4(-1)(1)}}{2}\\
	=&\dfrac{\pm\sqrt{4}}{2}=\begin{cases}
	x_1=+\dfrac{2}{2}=+1\\
	\\
	x_2=-\dfrac{2}{2}=-1\\
	\end{cases}	
	\intertext{Denominatore diverso da zero}
	x^2-4=&0\\
	x_{1,2}=&\dfrac{0\pm\sqrt{0-4(-4)(1)}}{2}\\
	=&\dfrac{\pm\sqrt{16}}{2}=\begin{cases}
	x_1=+\dfrac{4}{2}=+2\\
	\\
	x_2=-\dfrac{4}{2}=-2\\
	\end{cases}	
	\end{align*}
Unendo la disequazione con l'equazione otteniamo il grafico
\begin{center}
	\includestandalone[width=8.5cm]{quarto/DisSecGrado/FunzIrraEsempio4}
\end{center}
	Quindi il \PA{C.E.}	è
\[ x< -2\quad -2<x\leq -1\quad 1\leq x<2\quad 2<x \]
\end{esempiot}
\section{Irrazionale dispari intera}
\subsection{Definizione}
\begin{definizionet}{Irrazionale dispari intera}{}
	Una funzione $\funzione{f}{\R}{\R}$ è irrazionale diszzpari intera se l'incognita appartiene al radicando di una radice di indice dispari e  non si trova nel denominatore di una frazione. \index{Funzione!irrazionale!dispari intera}
\end{definizionet}
\begin{esempiot}{Irrazionale dispari intera}{}
	\begin{align*}
	f(x)=&\sqrt[3]{x}+1&f(x)=&\sqrt[5]{x^2+3x+1}\\
	f(x)=&\sqrt[7]{x^3+3x}+x&f(x)=&\sqrt[13]{x^2+3x+1}\\
	\end{align*}
\end{esempiot}
\section{Irrazionale dispari fratta}
\subsection{Definizione}
\begin{definizionet}{Irrazionale dispari fratta}{}
	Una funzione $\funzione{f}{\R}{\R}$ è irrazionale dispari intera se l'incognita appartiene al radicando di una radice di indice dispari e   si trova nel denominatore di una frazione. \index{Funzione!irrazionale!dispari fratta}
\end{definizionet}
\begin{esempiot}{Irrazionale dispari fratta}{}
	\begin{align*}
	f(x)=&\frac{1}{\sqrt[3]{x}}+1&f(x)=&\sqrt[3]{\frac{x-2}{x^2+2}}\\
	f(x)=&\frac{\sqrt[5]{x^2+1}}{x}&f(x)=&\frac{x}{\sqrt[3]{x^2-3}}\\
	\end{align*}
\end{esempiot}
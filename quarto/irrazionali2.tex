%\section{Funzioni irrazionali}
%\begin{esempiot}{Irrazionale *}{}
%	Consideriamo la funzione\[f(x)=\sqrt{x}\]
%\end{esempiot}
%\begin{enumerate}[noitemsep]
%	\litem{Classificazione}la funzione è una funzione irrazionale di indice pari. 
%	\litem{Dominio}una funzione irrazionale di indice pari esiste se il radicando è maggiore o uguale a zero. Quindi per determinare il dominio bisogna risolvere la disequazione\[x\geq0\]Otteniamo il~\cref{fig:Dirrazionale1} Il dominio della funzione è $\R-\lbrace x<0\rbrace $.
%	\litem{Positività}la funzione irrazionale è sempre positiva dove esiste.
%	\litem{Intersezioni assi}l'intersezione con l'asse $x$ si ottiene
%	\begin{align*}
%	&\begin{cases}
%	y=0\\
%	y=\sqrt{x}
%	\end{cases}
%	\intertext{otteniamo, considerando il dominio}
%	&\begin{cases}
%	y=0\\
%	x=0
%	\end{cases}
%	\end{align*}
%	L'intersezione con asse $y$ già determinata
%	Quindi le intersezioni sono $A(0,0)$
%	\litem{Pari e dispari} $f(-x)=\sqrt{(-x)}\neq f(x)$ non è pari e visto che $f(-x)=\sqrt{(-x)}\neq -f(x)$ non è dispari
%\end{enumerate}
%La~\cref{exa:irrazio1} riassume quanto detto.
%\begin{figure}
%	\captionsetup{name=Grafico}
%	\centering
%	\includestandalone[width=8.5cm]{quarto/dominio/disequazione9}
%	\caption{Dominio funzione}
%	\label[graf]{fig:Dirrazionale1}
%	\end{figure}
%\begin{funzionet}{Irrazionale}{irrazio1}
%	\includestandalone[width=\textwidth]{quarto/dominio/irrazzio1}
%	\tcblower
%	\begin{itemize}
%		\item $y=\sqrt{x}$
%		\item Dominio $\R-\lbrace x<0\rbrace$
%		\item Codominio $\Rp$
%		\item Positività sempre dove è definita
%		\item Intersezione asse $x$ $A(0,0)$
%		\item Intersezione asse $y$ $A(0,0)$
%		\item La funzione non è ne pari ne dispari
%	\end{itemize}
%\end{funzionet}

%\begin{esempiot}{Irrazionale *}{}
%	Consideriamo la funzione\[f(x)=\sqrt[3]{x}\]
%\end{esempiot}
%\begin{enumerate}[noitemsep]
%	\litem{Classificazione}la funzione è una funzione irrazionale di indice dispari. 
%	\litem{Dominio} il dominio di una funzione irrazionale di indice dispari coincide con quello del radicando. Il dominio della funzione è $\R$.
%	\litem{Positività}la funzione irrazionale è positiva quando il radicando è positivo. Risolvendo la disequazione\[x\geq0\]Otteniamo il~\cref{fig:Dirrazionale2}.
%	\litem{Intersezioni assi}l'intersezione con l'asse $x$ si ottiene
%	\begin{align*}
%	&\begin{cases}
%	y=0\\
%	y=\sqrt[3]{x}
%	\end{cases}
%	\intertext{otteniamo}
%	&\begin{cases}
%	y=0\\
%	x=0
%	\end{cases}
%	\end{align*}
%	L'intersezione con asse $y$ già determinata
%	Quindi le intersezioni sono $A(0,0)$
%	\litem{Pari e dispari} $f(-x)=\sqrt[3]{(-x)}=f(-x)$ la funzione è pari.
%\end{enumerate}
%La~\cref{exa:irrazio2} riassume quanto detto.
%\begin{figure}
%	\captionsetup{name=Grafico}
%	\centering
%	\includestandalone[width=8.5cm]{quarto/dominio/disequazione9}
%	\caption{Segno funzione}
%	\label[graf]{fig:Dirrazionale2}
%\end{figure}
%\begin{funzionet}{Irrazionale}{irrazio2}
%	\includestandalone[width=\textwidth]{quarto/dominio/irrazzio2}
%	\tcblower
%	\begin{itemize}
%		\item $y=\sqrt[3]{x}$
%		\item Dominio $\R$
%		\item Codominio $\R$
%		\item Positività $x>0$
%		\item Intersezione asse $x$ $A(0,0)$
%		\item Intersezione asse $y$ $A(0,0)$
%		%		\item Decrescente per $-2<x$ 
%		%		\item Crescente per $x>2$
%		\item Asintoti verticali nessuno
%		\item Asintoti orizzontali nessuno*
%		\item  La funzione è pari
%	\end{itemize}
%\end{funzionet}

\begin{esempiot}{Irrazionale ***}{}
	Consideriamo la funzione\[f(x)=\sqrt{\dfrac{x^2-4x+3}{x^2-6x+8}}\]
\end{esempiot}
\begin{enumerate}[noitemsep]
	\litem{Classificazione}la funzione è una funzione irrazionale di indice pari. 
	\litem{Dominio} il dominio di una funzione irrazionale di indice pari coincide con i valori per cui il radicando esiste ed è positivo. Il radicando è una frazione per cui bisogna escludere i valori per  il denominatore si annulla. Il dominio della funzione è $\R$.
	\litem{Positività}la funzione irrazionale è positiva quando il radicando è positivo. Risolvendo la disequazione\[x\geq0\]Otteniamo il~\cref{fig:Dirrazionale2}.
	\litem{Intersezioni assi}l'intersezione con l'asse $x$ si ottiene
	\begin{align*}
	&\begin{cases}
	y=0\\
	y=\sqrt[3]{x}
	\end{cases}
	\intertext{otteniamo}
	&\begin{cases}
	y=0\\
	x=0
	\end{cases}
	\end{align*}
	L'intersezione con asse $y$ già determinata
	Quindi le intersezioni sono $A(0,0)$
	\litem{Pari e dispari} $f(-x)=\sqrt[3]{(-x)}=f(-x)$ la funzione è pari.
\end{enumerate}
La~\cref{exa:irrazio2} riassume quanto detto.
\begin{figure}
	\captionsetup{name=Grafico}
	\centering
	\includestandalone[width=8.5cm]{quarto/dominio/disequazione9}
	\caption{Segno funzione}
	\label[graf]{fig:Dirrazionale2}
\end{figure}
\begin{funzionet}{Irrazionale}{irrazio2}
	\includestandalone[width=\textwidth]{quarto/dominio/irrazzio2}
	\tcblower
	\begin{itemize}
		\item $y=\sqrt[3]{x}$
		\item Dominio $\R$
		\item Codominio $\R$
		\item Positività $x>0$
		\item Intersezione asse $x$ $A(0,0)$
		\item Intersezione asse $y$ $A(0,0)$
		%		\item Decrescente per $-2<x$ 
		%		\item Crescente per $x>2$
		\item Asintoti verticali nessuno
		\item Asintoti orizzontali nessuno*
		\item  La funzione è pari
	\end{itemize}
\end{funzionet}















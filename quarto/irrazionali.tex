\section{Funzioni irrazionali}
\begin{esempiot}{Irrazionale *}{}
	Consideriamo la funzione\[f(x)=\sqrt{x}\]
\end{esempiot}
\begin{enumerate}[noitemsep]
	\litem{Classificazione}la funzione è una funzione irrazionale di indice pari. 
	\litem{Dominio}una funzione irrazionale di indice pari esiste se il radicando è maggiore o uguale a zero. Quindi per determinare il dominio bisogna risolvere la disequazione\[x\geq0\]Otteniamo il~\cref{fig:Dirrazionale1} Il dominio della funzione è $\R-\lbrace x<0\rbrace $.
	\litem{Positività}la funzione irrazionale è sempre positiva dove esiste.
	\litem{Intersezioni assi}l'intersezione con l'asse $x$ si ottiene
	\begin{align*}
	&\begin{cases}
	y=0\\
	y=\sqrt{x}
	\end{cases}
	\intertext{otteniamo, considerando il dominio}
	&\begin{cases}
	y=0\\
	x=0
	\end{cases}
	\end{align*}
	L'intersezione con asse $y$ già determinata
	Quindi le intersezioni sono $A(0,0)$
	\litem{Pari e dispari} $f(-x)=\sqrt{(-x)}\neq f(x)$ non è pari e visto che $f(-x)=\sqrt{(-x)}\neq -f(x)$ non è dispari
\end{enumerate}
La~\cref{exa:irrazio1} riassume quanto detto.
\begin{figure}
	\captionsetup{name=Grafico}
	\centering
	\includestandalone[width=8.5cm]{quarto/dominio/disequazione9}
	\caption{Dominio funzione}
	\label[graf]{fig:Dirrazionale1}
	\end{figure}
\begin{funzionet}{Irrazionale}{irrazio1}
	\includestandalone[width=\textwidth]{quarto/dominio/irrazzio1}
	\tcblower
	\begin{itemize}
		\item $y=\sqrt{x}$
		\item Dominio $\R-\lbrace x<0\rbrace$
		\item Codominio $\Rpos$
		\item Positività sempre dove è definita
		\item Intersezione asse $x$ $A(0,0)$
		\item Intersezione asse $y$ $A(0,0)$
		\item La funzione non è ne pari ne dispari
	\end{itemize}
\end{funzionet}

\begin{esempiot}{Irrazionale *}{}
	Consideriamo la funzione\[f(x)=\sqrt[3]{x}\]
\end{esempiot}
\begin{enumerate}[noitemsep]
	\litem{Classificazione}la funzione è una funzione irrazionale di indice dispari. 
	\litem{Dominio} il dominio di una funzione irrazionale di indice dispari coincide con quello del radicando. Il dominio della funzione è $\R$.
	\litem{Positività}la funzione irrazionale è positiva quando il radicando è positivo. Risolvendo la disequazione\[x\geq0\]Otteniamo il~\cref{fig:Dirrazionale2}.
	\litem{Intersezioni assi}l'intersezione con l'asse $x$ si ottiene
	\begin{align*}
	&\begin{cases}
	y=0\\
	y=\sqrt[3]{x}
	\end{cases}
	\intertext{otteniamo}
	&\begin{cases}
	y=0\\
	x=0
	\end{cases}
	\end{align*}
	L'intersezione con asse $y$ già determinata
	Quindi le intersezioni sono $A(0,0)$
	\litem{Pari e dispari} $f(-x)=\sqrt[3]{(-x)}=f(-x)$ la funzione è pari.
\end{enumerate}
La~\cref{exa:irrazio2} riassume quanto detto.
\begin{figure}
	\captionsetup{name=Grafico}
	\centering
	\includestandalone[width=8.5cm]{quarto/dominio/disequazione9}
	\caption{Segno funzione}
	\label[graf]{fig:Dirrazionale2}
\end{figure}
\begin{funzionet}{Irrazionale}{irrazio2}
	\includestandalone[width=\textwidth]{quarto/dominio/irrazzio2}
	\tcblower
	\begin{itemize}
		\item $y=\sqrt[3]{x}$
		\item Dominio $\R$
		\item Codominio $\R$
		\item Positività $x>0$
		\item Intersezione asse $x$ $A(0,0)$
		\item Intersezione asse $y$ $A(0,0)$
		%		\item Decrescente per $-2<x$ 
		%		\item Crescente per $x>2$
		\item Asintoti verticali nessuno
		\item Asintoti orizzontali nessuno*
		\item  La funzione è pari
	\end{itemize}
\end{funzionet}
\begin{esempiot}{Irrazionale **}{}
	Consideriamo la funzione\[f(x)=\sqrt{x^2-4}\]
\end{esempiot}
\begin{enumerate}[noitemsep]
	\litem{Classificazione}la funzione è una funzione irrazionale di indice pari. 
	\litem{Dominio}una funzione irrazionale di indice pari esiste se il radicando è maggiore o uguale a zero. Quindi per determinare il dominio bisogna risolvere la disequazione\[x^2-4\geq0\] Dato che l'equazione associata ha per soluzioni \[x=\pm2\] otteniamo il~\cref{fig:Dirrazionale3} Il dominio della funzione è $\R-\lbrace x\leq -2\quad x\geq 2\rbrace $.
	\litem{Positività}la funzione irrazionale è sempre positiva dove esiste.
	\litem{Intersezioni assi}l'intersezione con l'asse $x$ si ottiene
	\begin{align*}
	&\begin{cases}
	y=0\\
	y=\sqrt{x^2-4}
	\end{cases}
	\intertext{otteniamo, considerando il dominio}
	&\begin{cases}
	y=0\\
	x^2-4=0
	\end{cases}
	\intertext{otteniamo}
	&\begin{cases}
	y=0\\
	x=2
	\end{cases}&\begin{cases}
	y=0\\
	x=-2
	\end{cases}
	\end{align*}
	L'intersezione con asse $y$ non esiste
	Quindi le intersezioni sono $A(-2,0)$, $B(2,0)$
	\litem{Pari e dispari} $f(-x)=\sqrt{(-x)^2-4}=\sqrt{x^2-4}=f(x)$ la funzione è pari.
\end{enumerate}
La~\cref{exa:irrazio3} riassume quanto detto.
\begin{figure}
	\captionsetup{name=Grafico}
	\centering
	\includestandalone[width=8.5cm]{quarto/dominio/disequazione5}
	\caption{Dominio funzione}
	\label[graf]{fig:Dirrazionale3}
\end{figure}
\begin{funzionet}{Irrazionale}{irrazio3}
	\includestandalone[width=\textwidth]{quarto/dominio/irrazzio3}
	\tcblower
	\begin{itemize}
		\item $y=\sqrt{x^2-4}$
		\item Dominio $\R-\lbrace-2<x<2\rbrace$
		\item Codominio $\Rpos$
		\item Positività $-2<x$ $x>2$
		\item Intersezione asse $x$ $(-2,0)$ $(2,0)$
		\item Intersezione asse $y$ nessuna
%		\item Decrescente per $-2<x$ 
%		\item Crescente per $x>2$
		\item Asintoti verticali nessuno
		\item Asintoti orizzontali nessuno
		\item Funzione pari
	\end{itemize}
\end{funzionet}
\begin{esempiot}{Irrazionale fratta ***}{}
	Consideriamo la funzione\[f(x)=\sqrt{\dfrac{2x+1}{1-x}}\]
\end{esempiot}
\begin{enumerate}[noitemsep]
	\litem{Classificazione}la funzione è una funzione irrazionale fratta con indice pari. 
	\litem{Dominio}per determinare il dominio bisogna valutare l'esistenza della frazione e della radice. La frazione esiste se il denominatore è diverso da zero. Una funzione irrazionale di indice pari esiste se il radicando è maggiore o uguale a zero. Quindi per determinare il dominio bisogna risolvere l'equazione \[1-x=0\] e poi la disequazione\[\dfrac{2x+1}{1-x}\geq0\] L'equazione ha soluzione \[x=1\] Dato che la disequazione è frazionaria resta da risolvere le seguenti disequazioni \begin{align*}
	&1-x>0\\
	&2x+1\geq 0
	\intertext{quindi}
	&x<1\\
	&x\geq-\dfrac{1}{2}
	\end{align*}
	 otteniamo il~\cref{fig:irrazionale4}. Il dominio della funzione è $\R-\lbrace x\leq -\dfrac{1}{2}\vee x> 1\rbrace $.
	\litem{Positività}la funzione irrazionale è sempre positiva dove esiste.
	\litem{Intersezioni assi}l'intersezione con l'asse $x$ si ottiene
	\begin{align*}
	&\begin{cases}
	y=0\\
	y=\sqrt{\dfrac{2x+1}{1-x}}
	\end{cases}
	\intertext{otteniamo, considerando il dominio}
	&\begin{cases}
	y=0\\
	2x+1=0
	\end{cases}
	\intertext{otteniamo}
	&\begin{cases}
	y=0\\
	x=-\dfrac{1}{2}
	\end{cases}
	\end{align*}
	L'intersezione con asse $y$
	\begin{align*}
&\begin{cases}
x=0\\
y=\sqrt{\dfrac{2x+1}{1-x}}
\end{cases}\\
&\begin{cases}
x=0\\
y=1
\end{cases}
	\end{align*}
	Quindi le intersezioni sono $A(-\dfrac{1}{2},0)$, $B(0,1)$
	\litem{Pari e dispari} $f(-x)=\sqrt{\dfrac{2(-x)+1}{1-(-x)}}=\sqrt{\dfrac{-2x+1}{1+x}} \neq f(x)$ la funzione non è pari. La funzione non è dispari.
\end{enumerate}
La~\cref{exa:irrazio4} riassume quanto detto.
\begin{figure}
	\captionsetup{name=Grafico}
	\centering
	\includestandalone[width=8.5cm]{quarto/dominio/disequazione6}
	\caption{Dominio funzione}
	\label[graf]{fig:irrazionale4}
\end{figure}
\begin{funzionet}{Irrazionale fratta}{irrazio4}
	\includestandalone[width=\textwidth]{quarto/dominio/irrazzio4}
	\tcblower
	\begin{itemize}
		\item $y=\sqrt{\dfrac{2x+1}{1-x}}$
		\item Dominio $\R-\lbrace x<-\dfrac{1}{2}\vee x>1\rbrace$
		\item Codominio $\Rpos$
		\item Positività $-\dfrac{1}{2}<x<1$
		\item Intersezione asse $x$ $(-\dfrac{1}{2},0)$
		\item Intersezione asse $y$ $(0,1)$
		%\item Decrescente per $-2<x$ 
		%\item Crescente per $x>-\dfrac{1}{2}$
		\item Asintoti verticali uno
		\item Asintoti orizzontali nessuno
		\item Funzione ne pari ne dispari
	\end{itemize}
\end{funzionet}
\begin{esempiot}{Irrazionale fratta ***}{}
	Consideriamo la funzione\[f(x)=\sqrt{\dfrac{x^2-4x+3}{x^2-6x+8}}\]
\end{esempiot}
\begin{enumerate}[noitemsep]
	\litem{Classificazione}la funzione è una funzione irrazionale fratta con indice pari. 
	\litem{Dominio}per determinare il dominio bisogna valutare l'esistenza della frazione e della radice. La frazione esiste se il denominatore è diverso da zero. Una funzione irrazionale di indice pari esiste se il radicando è maggiore o uguale a zero. Quindi per determinare il dominio bisogna risolvere l'equazione \[x^2-6x+8=0\]
	\begin{align*}
x^2-6x+8=&0
	\intertext{otteniamo}
\begin{cases}
x_1\neq4\\
x_2\neq2
\end{cases}
	\end{align*}
	 la disequazione\[\dfrac{x^2-4x+3}{x^2-6x+8}\geq0\]  Dato che la disequazione è frazionaria resta da risolvere le seguenti disequazioni
	 \begin{align*}
	&x^2-4x+3\geq0\\
	&x^2-6x+8> 0
	\intertext{quindi}
	&x^2-4x+3=0\\
		\intertext{otteniamo}
	&\begin{cases}
	x_1=3\\
	x_2=1
	\end{cases}
	\end{align*}
	otteniamo il~\cref{fig:irrazionale5}. Il dominio della funzione è $\R-\lbrace 1\leq x<2\vee 3\leq x<4\rbrace $.
	\litem{Positività}la funzione irrazionale è sempre positiva dove esiste.
	\litem{Intersezioni assi}l'intersezione con l'asse $x$ si ottiene
	\begin{align*}
	&\begin{cases}
	y=0\\
	y=\sqrt{\dfrac{x^2-4x+3}{x^2-6x+8}}
	\end{cases}
	\intertext{otteniamo, considerando il dominio}
	&\begin{cases}
	y=0\\
	x^2-4x+3=0
	\end{cases}
	\intertext{otteniamo}
	&\begin{cases}
	y=0\\
	x=1
	\end{cases}&\begin{cases}
	y=0\\
	x=3
	\end{cases}
	\end{align*}
	L'intersezione con asse $y$
	\begin{align*}
	&\begin{cases}
	x=0\\
	y=\sqrt{\dfrac{x^2-4x+3}{x^2-6x+8}}
	\end{cases}\\
	&\begin{cases}
	x=0\\
	y=\sqrt{\dfrac{3}{8}}
	\end{cases}
	\end{align*}
	Quindi le intersezioni sono $A(1,0)$, $B(3,0)$ e $C(0,\sqrt{\dfrac{3}{8}})$
	\litem{Pari e dispari} $f(-x)=\sqrt{\dfrac{(-x)^2-4(-x)+3}{( -x)^2-6(-x)+8}}\neq f(x)$ la funzione non è pari. La funzione non è dispari.
\end{enumerate}
La~\cref{exa:irrazio5} riassume quanto detto.
\begin{figure}
	\captionsetup{name=Grafico}
	\centering
	\includestandalone[width=8.5cm]{quarto/dominio/disequazione10}
	\caption{Dominio funzione}
	\label[graf]{fig:irrazionale5}
\end{figure}
\begin{funzionet}{Irrazionale fratta}{irrazio5}
	\includestandalone[width=\textwidth]{quarto/dominio/irrazzio5}
	\tcblower
	\begin{itemize}
		\item $y=\sqrt{\dfrac{x^2-4x+3}{x^2-6x+8}}$
		\item Dominio $\R-\lbrace 1\leq x<2\vee 3\leq x<4\rbrace $
		\item Codominio $\Rpos$
		\item Positività dove è definita
		\item Intersezione asse $x$ $A(1,0)$, $B(3,0)$
		\item Intersezione asse $y$ $C(0,\sqrt{\dfrac{3}{8}})$
		%\item Decrescente per $-2<x$ 
		%\item Crescente per $x>-\dfrac{1}{2}$
		\item Asintoti verticali due
		\item Asintoti orizzontali uno
		\item Funzione ne pari ne dispari
	\end{itemize}
\end{funzionet}
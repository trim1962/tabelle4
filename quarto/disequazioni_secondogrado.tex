\chapter{Disequazioni di secondo grado}
\label{cha:Disequazionisecondogrado}
%\section{Disequazioni intere}
%Una disequazione di secondo grado\index{Disequazione!secondo grado!intera} è un'espressione del tipo
%\begin{equation}
%2x^2-x-1\geq 0\label{equ:DisSecondoGrado1}
%\end{equation} 
%in cui abbiamo un trinomio posto maggiore o uguale di zero. Per risolvere la disequazione  possiamo procedere in questo modo. Trasformiamo  il problema in un altro. Per far ciò  utilizziamo la relazione
%\begin{align}
%&ax^2+bx+c=a(x-x_{1})(x-x_{2})\label{DisTrinsecGrado0}\\
%&x_{1}=\dfrac{-b+\sqrt{b^2-4ac}}{2a}\label{DisTrinsecGrado1}\\
%&x_{2}=\dfrac{-b-\sqrt{b^2-4ac}}{2a}\label{DisTrinsecGrado2}
%\end{align}
%Quindi utilizzando le relazioni\nobs\vrefrange{DisTrinsecGrado0}{DisTrinsecGrado2} possiamo scrivere 
%\begin{align}
%&2x^2-x-1\\
%&x_{1}=\dfrac{1+\sqrt{9}}{4}\notag
%&x_{1}=\dfrac{1-\sqrt{9}}{4}\notag\\
%&x_{1}=\dfrac{1+3}{4}\notag
%&x_{1}=\dfrac{1-3}{4}\notag\\
%&x_{1}=\dfrac{4}{4}=1\notag
%&x_{1}=-\dfrac{2}{4}=-\dfrac{1}{2}\notag\\
%&2x^2-x-1=2(x-1)(x+\dfrac{1}{2})\label{DisTrinsecGradoEs1}
%\end{align}
%La relazione\nobs\vref{DisTrinsecGradoEs1} trasforma un trinomio di secondo grado nel prodotto di due binomi e un numero. Quindi la disequazione\nobs\vref{equ:DisSecondoGrado1} diventa 
%\begin{equation}
%2(x-1)(x+\dfrac{1}{2})\geq 0\label{equ:DisSecondogrado1a}
%\end{equation} 
%Ottengo il grafico\nobs\vref{fig:esempioDisSecGrad1}. Il disegno mostra tre righe una per ogni fattore del prodotto\nobs\vref{equ:DisSecondogrado1a}. Nel particolare il trinomio è positivo per $x\leq-\dfrac{1}{2}$ o $x\geq 1$ negativo per $-\dfrac{1}{2}\leq x\leq 1$
%
%Consideriamo un altro esempio
%\begin{equation}
%-3x^2+4x-1\geq 0\label{equ:DisSecondoGrado2}
%\end{equation} 
%anche qui abbiamo un trinomio posto maggiore o uguale di zero. Per risolvere la disequazione  procediamo come prima  Trasformando  il problema in un altro. 
%Quindi utilizzando le relazioni\nobs\vrefrange{DisTrinsecGrado0}{DisTrinsecGrado2} possiamo scrivere: 
%\begin{align}
%&-3x^2+4x-1\\
%&x_{1}=\dfrac{-4+\sqrt{4}}{-6}\notag
%&x_{1}=\dfrac{-4-\sqrt{4}}{-6}\notag\\
%&x_{1}=\dfrac{-4+2}{-6}\notag
%&x_{1}=\dfrac{-4-2}{-6}\notag\\
%&x_{1}=\dfrac{-2}{-6}=\dfrac{1}{3}\notag
%&x_{1}=-\dfrac{-6}{-6}=1\notag\\
%&-3x^2+4x-1=-3(x-1)(x-\dfrac{1}{3})\label{DisTrinsecGradoEs2}
%\end{align}
%Anche questa volta relazione\nobs\vref{DisTrinsecGradoEs2} trasforma il trinomio di secondo grado nel prodotto di due binomi e un numero. Quindi la disequazione\nobs\vref{equ:DisSecondoGrado2} diventa 
%\begin{equation}
%-3(x-1)(x-\dfrac{1}{3})\geq 0\label{equ:DisSecondogrado2a}
%\end{equation} 
%Ottengo il grafico\nobs\vref{fig:esempioDisSecGrad2}. Il disegno mostra tre righe una per ogni fattore del prodotto\nobs\vref{equ:DisSecondogrado2a}. Nel particolare il trinomio è negativo per $x\leq\dfrac{1}{3}$ o $x\geq 1$, positivo per $-\dfrac{1}{3}\leq x\leq 1$
%\begin{figure}
%	\centering
%		\begin{subfigure}[b]{.4\linewidth}
%			\centering
%			\includestandalone[width=\textwidth]{quarto/DisSecGrado/DisSecGradoesempio1}
%			\caption{Esempio 1}
%			\label{fig:esempioDisSecGrad1}
%		\end{subfigure}%
%	\centering
%	\quad
%	\begin{subfigure}[b]{.4\linewidth}
%			\centering
%			\includestandalone[width=\textwidth]{quarto/DisSecGrado/DisSecGradoesempio2}
%			\caption{Esempio 2}
%			\label{fig:esempioDisSecGrad2}
%	\end{subfigure}%
%\caption{$\Delta>0$}
%\label{fig:DeltaMagZeroEsempio1}
%\end{figure}
%
%La figura\nobs\vref{fig:DeltaMagZeroEsempio1} permette di confrontare i due grafici. Questi sono praticamente identici. \'{E} la terza riga, continua nel primo esempio, tratteggiata nel secondo, che fa la differenza. Guardando i due grafici, vediamo che all'esterno dell'intervallo formato dalle due soluzioni, il grafico ha lo stesso segno del coefficiente $a$. Mentre nello spazio compreso fra le due soluzioni, il segno è opposto a quello di $a$. Otteniamo due grafici come\nobs\vrefrange{graf:dis2GDeltaMagZa2x}{graf:dis2GDeltaMagZb2x}. Questo è riassunto graficamente dalla prima riga della tabella\nobs\vref{tab:segnodisequazioni2grado}
%\begin{figure}
%	\begin{subfigure}[b]{.5\linewidth}
%		\centering
%\includestandalone[width=\textwidth]{quarto/DisSecGrado/DeltaMaggioreDiZeroAmaggioreDizero}
%		\caption{$\Delta>0$ $a>0$}\label{graf:dis2GDeltaMagZa2x}
%	\end{subfigure}%
%\quad
%	\begin{subfigure}[b]{.5\linewidth}
%		\centering
%	\includestandalone[width=\textwidth]{quarto/DisSecGrado/DeltaMaggioreDiZeroAminoreDizero}
%		\caption{$\Delta>0$ $a<0$}\label{graf:dis2GDeltaMagZb2x}
%	\end{subfigure}
%\vskip .8cm
%	\begin{subfigure}[b]{.5\linewidth}
%		\centering
%		\includestandalone[width=\textwidth]{quarto/DisSecGrado/DeltaUgualeaZeroAmaggioreDizero}
%		\caption{$\Delta=0$ $a>0$}\label{graf:dis2GDeltaUguaZa2x}
%			\end{subfigure}%
%\quad
%	\begin{subfigure}[b]{.5\linewidth}
%		\centering
%		\includestandalone[width=\textwidth]{quarto/DisSecGrado/DeltaUgualeaZeroAminoreDizero}
%		\caption{$\Delta=0$ $a<0$}\label{graf:dis2GDeltaUguaZb2x}
%	\end{subfigure}
%\vskip .8cm
%\begin{subfigure}[b]{.5\linewidth}
%	\centering
%		\includestandalone[width=\textwidth]{quarto/DisSecGrado/DeltaMinoreZeroAmaggioreDizero}
%	\caption{$\Delta<0$ $a>0$}\label{graf:dis2GDeltaMinorZa2x}
%\end{subfigure}%
%\quad
%\begin{subfigure}[b]{.5\linewidth}
%	\centering
%\includestandalone[width=\textwidth]{quarto/DisSecGrado/DeltaMinoreZeroAminoreDizero}
%	\caption{$\Delta<0$ $a<0$}\label{graf:dis2GDeltaMinorZb2x}
%\end{subfigure}
%	\caption{Grafici disequazione di secondo grado}
%\end{figure}
%\begin{table}
%	\begin{tabular}{@{}m{1cm}m{7.8cm}m{7.8cm}}
%	%	\toprule
%		& \centering $a>0$ & \centering$a<0$\tabularnewline
%\centering$\Delta>0$ &\tabincludestandalone[width=7.5cm]{quarto/DisSecGrado/DeltaMaggioreDiZeroAmaggioreDizero}  & \vskip .8cm 	\tabincludestandalone[width=7.5cm]{quarto/DisSecGrado/DeltaMaggioreDiZeroAminoreDizero} \\[1cm] 
%		\centering$\Delta=0$ & 	\tabincludestandalone[width=7.5cm]{quarto/DisSecGrado/DeltaUgualeaZeroAmaggioreDizero} & \vskip .8cm \tabincludestandalone[width=7.5cm]{quarto/DisSecGrado/DeltaUgualeaZeroAminoreDizero}\\[1cm] 
%		\centering$\Delta<0$ & \tabincludestandalone[width=7.5cm]{quarto/DisSecGrado/DeltaMinoreZeroAmaggioreDizero} &\vskip .8cm \tabincludestandalone[width=7.5cm]{quarto/DisSecGrado/DeltaMinoreZeroAminoreDizero}\\[1cm] 
%	%	\bottomrule
%	\end{tabular}
%	\caption{Segno disequazioni secondo grado}
%\end{table}
%\begin{table}
%	\centering
%	\begin{tikzpicture}
%	\tkzTabInit[color,lgt=5,espcl=3]%
%	{$x$ / .8,$\Delta>0$\\ Il segno di\\ $ax^2+bx+c$ /2}%
%	{$-\infty$,$x_1$,$x_2$,$+\infty$}%
%	\tkzTabLine{ , \genfrac{}{}{0pt}{0}{\text{segno di}}{a}, z
%		, \genfrac{}{}{0pt}{0}{\text{segno}}{\text{opposto di}\ a}, z
%		, \genfrac{}{}{0pt}{0}{\text{segno di}}{a}, }
%	\end{tikzpicture}\\
%	\begin{tikzpicture}
%	\tkzTabInit[color,lgt=5,espcl=3]%
%	{$x$ / .8, $\Delta=0$\\ Il segno di\\ $ax^2+bx+c$ / 2}%
%	{$-\infty$,$x_1$,$+\infty$}%
%	\tkzTabLine{ , \genfrac{}{}{0pt}{0}{\text{segno di}}{ a} , z
%		, \genfrac{}{}{0pt}{0}{\text{segno di}}{a}, }
%	\end{tikzpicture}\\
%	\begin{tikzpicture}
%	\tkzTabInit[color,lgt=5,espcl=5]%
%	{$x$/.8,$\Delta<0$\\ Il segno di\\ $ax^2+bx+c$/2}%
%	{$-\infty$,$+\infty$}%
%	\tkzTabLine{ , \genfrac{}{}{0pt}{0}{\text{segno di}}{ a}, }
%	\end{tikzpicture}
%	\caption{Segno disequazione di secondo grado}
%	\label{tab:segnodisequazioni2grado}
%\end{table}
%
%Consideriamo una disequazione del tipo
%\begin{equation}
%2x^2-4x+2\geq 0\label{equ:DisSecondoGrado3}
%\end{equation} 
%come negli esempi precedenti abbiamo un trinomio di secondo grado  posto maggiore o uguale di zero. Anche qui trasformiamo  il problema in un altro. Per far ciò  utilizziamo le relazioni\nobs\vrefrange{DisTrinsecGrado0}{DisTrinsecGrado2}.
%Possiamo scrivere:
%\begin{align}
%&2x^2-4x+2\\
%&x_{1}=\dfrac{4+\sqrt{0}}{4}\notag
%&x_{1}=\dfrac{4-\sqrt{0}}{4}\notag\\
%&x_{1}=\dfrac{4}{4}=1\notag
%&x_{1}=\dfrac{4}{4}=1\notag\\
%&2x^2-4x+2=2(x-1)(x-1)=2(x-1)^2\label{DisTrinsecGradoEs3}
%\end{align}
%La relazione\nobs\vref{DisTrinsecGradoEs1} trasforma un trinomio di secondo grado nel prodotto di un binomio al quadrato e un numero. Quindi la disequazione\nobs\vref{equ:DisSecondoGrado3} diventa 
%\begin{equation}
%2(x-1)^2\geq 0\label{equ:DisSecondogrado3a}
%\end{equation} 
%Ottengo il grafico\nobs\vref{fig:esempioDisSecGrad3}. Il disegno mostra tre righe una per ogni fattore del prodotto\nobs\vref{equ:DisSecondogrado3a}. Nel particolare il trinomio è positivo per $x1$ e $x>1$, vale zero  per $x=1$
%\begin{figure}
%	\centering
%	\begin{subfigure}[b]{.4\linewidth}
%		\centering
%		\includestandalone[width=\textwidth]{quarto/DisSecGrado/DisSecGradoesempio3}
%		\caption{Esempio 3}
%		\label{fig:esempioDisSecGrad3}
%	\end{subfigure}%
%	\quad\centering
%	\begin{subfigure}[b]{.4\linewidth}
%		\centering
%		\includestandalone[width=\textwidth]{quarto/DisSecGrado/DisSecGradoesempio4}
%		\caption{Esempio 4}
%		\label{fig:esempioDisSecGrad4}
%	\end{subfigure}%
%	\caption{$\Delta=0$}
%	\label{fig:DeltaUguZeroEsempio2}
%\end{figure}
%
%Continuiamo con gli esempi
%\begin{equation}
%-3x^2+12x-12\geq 0\label{equ:DisSecondoGrado4}
%\end{equation} 
%come in precedenza abbiamo un trinomio di secondo grado  posto maggiore o uguale di zero.  Utilizziamo anche qui le relazioni\nobs\vrefrange{DisTrinsecGrado0}{DisTrinsecGrado2}.
%Possiamo scrivere:
%\begin{align}
%&-3x^2+12x-12\\
%&x_{1}=\dfrac{-12+\sqrt{0}}{-6}\notag
%&x_{1}=\dfrac{-12-\sqrt{0}}{-6}\notag\\
%&x_{1}=\dfrac{-12}{-6}=2\notag
%&x_{1}=\dfrac{-12}{-6}=2\notag\\
%&-3x^2+12x-12=-3(x-2)(x-2)=-3(x-2)^2\label{DisTrinsecGradoEs4}
%\end{align}
%La relazione\nobs\vref{DisTrinsecGradoEs1} trasforma un trinomio di secondo grado nel prodotto di un binomio al quadrato e un numero. Quindi la disequazione\nobs\vref{equ:DisSecondoGrado4} diventa 
%\begin{equation}
%-3(x-2)^2\geq 0\label{equ:DisSecondogrado4a}
%\end{equation} 
%Ottengo il grafico\nobs\vref{fig:esempioDisSecGrad4}. Il disegno mostra tre righe una per ogni fattore del prodotto\nobs\vref{equ:DisSecondogrado4a}. Nel particolare il trinomio è negativo per $x<2$ e $x>2$, vale zero  per $x=2$
%
%Confrontiamo i due grafici tramite la figura\nobs\vref{fig:DeltaUguZeroEsempio2}. Anche  la terza  riga, continua nel primo esempio, tratteggiata nel secondo, fa la differenza. Guardando i due grafici, vediamo che il grafico ha lo stesso segno del coefficiente $a$ tranne per $x=x_1$ in cui vale zero.  come\nobs\vrefrange{graf:dis2GDeltaUguaZa2x}{graf:dis2GDeltaUguaZb2x}. Questo è riassunto graficamente dalla seconda riga della tabella\nobs\vref{tab:segnodisequazioni2grado}
%
%Nella prima coppia di esempi avevamo due soluzioni distinte, nella seconda le soluzioni erano coincidenti. Resta da considerare quando le soluzioni non esistono.
%
%In questo caso il discriminate dell'equazione è un numero minore di zero. Si può dimostrare con qualche calcolo in più che i grafici sono come quelli delle figure\nobs\vrefrange{graf:dis2GDeltaMinorZa2x}{graf:dis2GDeltaMinorZb2x}. In questo caso il grafico ha sempre lo stesso segno di $a$  nella terza riga della tabella\nobs\vref{tab:segnodisequazioni2grado}
%\begin{table}
%	\centering
%	 \begin{tabular}{@{}cc>{\centering}m{6.5cm}>{\centering}m{6.5cm}}
%	 	&  & $a>0$ & \vskip .2cm $a<0$ \tabularnewline[0.5cm] 
%	 	&  & 	\tabincludestandalone[width=6.5cm]{quarto/DisSecGrado/DeltaMaggioreDiZeroAmaggioreDizero}  & \vskip .2cm	\tabincludestandalone[width=6.5cm]{quarto/DisSecGrado/DeltaMaggioreDiZeroAminoreDizero} \tabularnewline[0.5cm] 
%	 	\multirow{4}{1cm}{$\Delta>0$}	& $ax^2+bx+c\geq 0$ & $x\leq x_1$ e $x\geq x_2$  & $x_1\leq x \leq x_2$ \tabularnewline  
%	 	& $ax^2+bx+c > 0$ &$x< x_1$ e $x>x_2$  & $x_1< x < x_2$ \tabularnewline
%	 	& $ax^2+bx+c\leq 0$ & $x_1\leq x \leq x_2$ & $x\leq x_1$ e $x\geq x_2$ \tabularnewline  
%	 	& $ax^2+bx+c< 0$ & $x_1< x < x_2$ & $x< x_1$ e $x>x_2$ \tabularnewline
%	 	&&&\tabularnewline
%	 	&  & 	\tabincludestandalone[width=6.5cm]{quarto/DisSecGrado/DeltaUgualeaZeroAmaggioreDizero} &\vskip .2cm  \tabincludestandalone[width=6.5cm]{quarto/DisSecGrado/DeltaUgualeaZeroAminoreDizero}\tabularnewline[0.5cm] 
%	 	\multirow{4}{1cm}{$\Delta=0$}	& $ax^2+bx+c\geq 0$ & Sempre & $x=x_1$ \tabularnewline  
%	 	& $ax^2+bx+c > 0$ & Sempre $x\neq x_1$ & Mai \tabularnewline
%	 	& $ax^2+bx+c\leq 0$ & $x=x_1 $  & Sempre \tabularnewline  
%	 	& $ax^2+bx+c< 0$ & Mai & Sempre $x\neq x_1$ \tabularnewline  
%	 		&&&\tabularnewline
%	 	&  & 	\tabincludestandalone[width=6.5cm]{quarto/DisSecGrado/DeltaMinoreZeroAmaggioreDizero} &\vskip .2cm \tabincludestandalone[width=6.5cm]{quarto/DisSecGrado/DeltaMinoreZeroAminoreDizero}\tabularnewline[0.5cm] 
%	 	\multirow{4}{1cm}{$\Delta<0$}	& $ax^2+bx+c\geq 0$ & Sempre. Uguale a zero mai & Mai \tabularnewline  
%	 	& $ax^2+bx+c > 0$ & Sempre & Mai \tabularnewline
%	 	& $ax^2+bx+c\leq 0$ & Mai & Sempre. Uguale a zero mai \tabularnewline  
%	 	& $ax^2+bx+c< 0$ & Mai & Sempre \tabularnewline  
%	 \end{tabular} 
%	\caption{Soluzioni disequazioni secondo grado}
%	\label{tab:SoluzioniDisequazioniSecondoGrado}
%\end{table}
\section{Disequazioni di secondo grado intere}
\begin{definizionet}{Disequazione intera in forma normale}{}
Una disequazione di secondo grado\index{Disequazione!secondo grado} intera è in forma normale se \[ax^2+bx+c\;\begin{cases}
>\\
<\\
\leq\\
\geq
\end{cases} 0\; \text{con}\; a\neq 0\]
\end{definizionet}
\begin{osservazionet}{}{}
Le seguenti disequazioni sono tutte in forma normale\index{Disequazione!forma normale}
\begin{align*}
&2x^2+3x+2>0\\
&3x^2+2\leq0\\
&-x^2+4x\geq0\\
&5x^2\geq0
\end{align*}
\end{osservazionet} 
\begin{esempiot}{Delta maggiore di zero $a$ maggiore di zero }{DeltaMaggiorediZeroamaggiore}
	Consideriamo la disequazione in forma normale
	\begin{align*}
	&x^2+x-12\geq0
	\intertext{ad essa è associata l'equazione}
	&x^2+x-12=0\\
	&x_{1,2}=\dfrac{-1\pm\sqrt{1+48}}{2}=\dfrac{-1\pm7}{2}=
	\begin{cases}
	x_1=+3\\x_2=-4
	\end{cases}
	\end{align*} 
Possiamo associare alla disequazione una parabola del tipo \[y=x^2+x-12\]
la parabola, visti i punti di intersezione calcolati prima e il coefficiente $a$ positivo, ha come grafico la figura~\vref{fig:DeltaMaggioreZeroEsempio1}. La disequazione può essere vista come \[y=x^2+x-12\geq 0 \] Guardando il grafico è evidente che $y$ è positiva per valori  di $x<-4$ e per valori di $x>3$. Inoltre è negativa per $-4<x<3$ mentre per $x=-4$ e per $x=3$ $y$ vale zero. Possiamo riportare quanto detto in precedenza nel grafico della figura~\vref{fig:DeltaMaggioreZeroGraficoEsempio1}. Possiamo risolvere la disequazione, e visto che richiede quando è maggiore o uguale a zero, la soluzione è per $x\leq -4$ o per $x\geq3$ 
\end{esempiot}
\begin{figure}
	\centering
	\includestandalone[width=8.5cm]{quarto/DisSecGrado/parabolaDeltapiuApiu}
	\caption{$\Delta>0$ $a>0$ }
	\label{fig:DeltaMaggioreZeroEsempio1}
\end{figure}
\begin{figure}
	\centering
	\includestandalone[width=8.5cm]{quarto/DisSecGrado/parabolaDeltapiuApiuGrafico}
	\caption{Segno $\Delta>0$ $a>0$}
	\label{fig:DeltaMaggioreZeroGraficoEsempio1}
\end{figure}
\begin{esempiot}{Delta maggiore di zero $a$ minore di zero}{}
	Consideriamo la disequazione in forma normale
	\begin{align*}
	&-2x^2+5x+7<0
	\intertext{ad essa è associata l'equazione}
	&-2x^2+5x+7=0\\
	&x_{1,2}=\dfrac{-5\pm\sqrt{25+56}}{-4}=\dfrac{-5\pm9}{-4}=
	\begin{cases}
	x_1=+\dfrac{7}{2}\\
	\\x_2=-1
	\end{cases}
	\end{align*} 
Come per l'esempio precedente disegniamo il grafico della parabola \[y=-2x^2+5x+7\] Questa parabola, per quanto calcolato prima e per $a<0$, ha il grafico è come quello della figura~\vref{fig:DeltaMaggioreZeroEsempio2} 

Possiamo dire che la parabola è negativa per valori di $x$ minori di meno uno e maggiori di $\dfrac{7}{2}$ mentre è positiva per valori di $x$ compresi tra meno uno e  $\dfrac{7}{2}$. Riportando graficamente quanto detto otteniamo il grafico~\vref{fig:DeltaMaggioreZeroGraficoEsempio2}. La soluzione è  $x\leq -4$ o per $x\geq3$ 
\end{esempiot}
\begin{figure}
	\centering 
	\includestandalone[width=7.5cm]{quarto/DisSecGrado/parabolaDeltapiuAmeno}
	\caption{$\Delta>0$ $a<0$}
	\label{fig:DeltaMaggioreZeroEsempio2}
\end{figure}
\begin{figure}
	\centering
	\includestandalone[width=8.5cm]{quarto/DisSecGrado/parabolaDeltapiuAmenoGrafico}
	\caption{Segno $\Delta>0$ $a<0$}
	\label{fig:DeltaMaggioreZeroGraficoEsempio2}
\end{figure}
\begin{esempiot}{Delta uguale a zero $a$ maggiore di zero}{DeltaUgualeaZeroaMaggiore}
	Consideriamo la disequazione in forma normale
\begin{align*}
&x^2-2x+1<0
\intertext{ad essa è associata l'equazione}
&x^2-2x+1=0\\
&x_{1,2}=\dfrac{2\pm\sqrt{4-4}}{2}=\dfrac{2}{2}=1
\end{align*} 
Possiamo associare alla disequazione una parabola del tipo \[y=x^2-2x+1\]
la parabola, visti il punto di intersezione calcolato prima e il coefficiente $a$ positivo, ha come grafico la figura~\vref{fig:DeltaUgualeaZeroEsempio3}

Dal grafico è evidente che la parabola è positiva per qualunque valore di $x$, tranne che per $x=1$ in cui $y=0$. Si ottiene un grafico come quello della figura~\vref{fig:DeltaUgualeaZeroGraficoEsempio3}. La disequazione richiedendo quali sono i valori di $x$ per cui il trinomio è negativo, non ha soluzione. 
\end{esempiot}
\begin{figure}
	\centering
	\includestandalone[width=8.5cm]{quarto/DisSecGrado/parabolaDeltazeroApiuGrafico}
	\caption{Segno $\Delta=0$ $a>0$}
	\label{fig:DeltaUgualeaZeroGraficoEsempio3}
\end{figure}
\begin{figure}
	\centering 
	\includestandalone[width=7.5cm]{quarto/DisSecGrado/parabolaDeltazeroApiu}
	\caption{$\Delta=0$ $a>0$}
	\label{fig:DeltaUgualeaZeroEsempio3}
\end{figure}
\begin{esempiot}{Delta uguale a zero $a$ minore di zero}{}
	Considero la disequazione
	\begin{align*}
	&-x^2+4x-4<0
	\intertext{ad essa è associata l'equazione}
	&-x^2+4x-4=0\\
	&x_{1,2}=\dfrac{-4\pm\sqrt{16-16}}{-2}=\dfrac{-4}{-2}=2
	\end{align*} 
Possiamo associare alla disequazione una parabola del tipo \[y=-x^2+4x-4\]
la parabola, visti il punto di intersezione calcolato prima e il coefficiente $a$ negativo, ha come grafico la figura~\ref{fig:DeltaUgualeaZeroEsempio4}. Il trinomio ha per segno come indicato nel grafico~\vref{fig:DeltaUgualeaZeroGraficoEsempio4}. La soluzione sarà sempre verificata con $x\neq 2$
\end{esempiot}
\begin{figure}
	\centering
	\includestandalone[width=8.5cm]{quarto/DisSecGrado/parabolaDeltazeroAmenoGrafico}
	\caption{Segno $\Delta=0$ $a<0$}
	\label{fig:DeltaUgualeaZeroGraficoEsempio4}
\end{figure}
\begin{figure}
	\centering 
	\includestandalone[width=7.5cm]{quarto/DisSecGrado/parabolaDeltazeroAmeno}
	\caption{$\Delta=0$ $a<0$}
	\label{fig:DeltaUgualeaZeroEsempio4}
\end{figure}
\begin{esempiot}{Delta minore di zero a maggiore di zero}{}
	\begin{align*}
	&x^2+x+1<0
	\intertext{ad essa è associata l'equazione}
	&x^2+x+1=0\\
	&x_{1,2}=\dfrac{-1\pm\sqrt{1-4}}{2}
	\intertext{non ha soluzione}
	\end{align*} 
	Per questo e dato che $a>0$ il grafico della parabola è come quello della figura~\vref{fig:DeltaminoreZeroEsempio5}. Il grafico della disequazione è il grafico~\vref{fig:DeltaMinoreZeroGraficoEsempio5}. Quindi visto che è sempre positivo la disequazione non è mai verificata.
\end{esempiot}
\begin{figure}
	\centering
	\includestandalone[width=8.5cm]{quarto/DisSecGrado/parabolaDeltamenoApiuGrafico}
	\caption{Segno $\Delta<0$ $a>0$}
	\label{fig:DeltaMinoreZeroGraficoEsempio5}
\end{figure}
\begin{figure}
	\centering 
	\includestandalone[width=7.5cm]{quarto/DisSecGrado/parabolaDeltamenoApiu}
	\caption{$\Delta<0$ $a>0$}
	\label{fig:DeltaminoreZeroEsempio5}
\end{figure}
\begin{esempiot}{Delta minore di zero a minore di zero}{}
	\begin{align*}
&-2x^2-1<0
\intertext{ad essa è associata l'equazione}
&-2x^2-1=0\\
&x_{1,2}=\dfrac{0\pm\sqrt{0-8}}{2}
\intertext{non ha soluzione}
\end{align*} Per questo e dato che $a<0$ il grafico della parabola è come quello della figura~\vref{fig:DeltaminoreZeroEsempio6}. Il grafico corrispondente è quello della figura~\vref{fig:DeltaMinoreZeroGraficoEsempio6}
\end{esempiot}
\begin{figure}
	\centering
	\includestandalone[width=8.5cm]{quarto/DisSecGrado/parabolaDeltamenoAmenoGrafico}
	\caption{Segno $\Delta<0$ $a<0$}
	\label{fig:DeltaMinoreZeroGraficoEsempio6}
\end{figure}
\begin{figure}
	\centering 
	\includestandalone[width=7.5cm]{quarto/DisSecGrado/parabolaDeltamenoAmeno}
	\caption{$\Delta<0$ $a<0$}
	\label{fig:DeltaminoreZeroEsempio6}
\end{figure}

Ricapitolando gli esercizi precedenti, ad ogni equazione sono associati un numero detto delta\index{Equazione!delta} \[\Delta=b^2-4ac\] ed un'equazione\index{Equazione!secondo grado} di secondo grado\[ax^2+bx+c=0\]  Per risolvere una disequazione di secondo grado intera procedo come segue
\begin{enumerate}
	\item Metto la disequazione in forma normale
	\item Memorizzo il segno di $a$
	\item Risolvo l'equazione corrispondente $ax^"+bx+c=0$ tramite $x_{1,2}=\dfrac{-b\pm\sqrt{\Delta}}{2a}$. Avremo tre casi
	\begin{description}
		\item[3a] l'equazione ha due soluzioni distinte $x_1\neq x_2$, $\Delta>0$
		\item[3b] l'equazione ha due soluzioni coincidenti $x_1= x_2$, $\Delta=0$
		\item[3c] l'equazione non ha soluzioni $\Delta<0$
	\end{description}
\item Disegno il grafico. Dal numero delle soluzioni ho tre casi
\begin{description}
	\item[Caso 3a] Soluzioni distinte\begin{enumerate}
		\item riporto le soluzioni sull'asse $x$, ordinandole dalla minore  alla maggiore.
		\item traccio due segmenti verticali di stessa lunghezza per ogni soluzione
		\item fuori dello spazio tra le due soluzioni, il grafico ha lo stesso segno di $a$. Quindi se $a$ è positiva fuori traccio due linee continue. Se $a$ è negativa fuori  disegno due linee tratteggiate. Tra le due soluzioni traccio l'opposto di quello che c'è fuori. Si dice che i grafici sono DICE cioè \textbf{C}oncordi \textbf{I}nterni \textbf{C}oncordi \textbf{E}sterni
	\end{enumerate}
\item[Caso 3b] Soluzioni coincidenti
\begin{enumerate}
	\item riporto la soluzione sull'asse $x$.
	\item traccio un segmento verticale per la soluzione
	\item  se $a$ è positiva traccio una linea continua. Se $a$ è negativa  traccio una linea tratteggiata. Il grafico è concorde tranne in un punto.
\end{enumerate} 
\item[Caso 3c] Nessuna soluzione
\begin{enumerate}
	\item se $a$ è positiva traccio una linea continua, se $a$ è negativa traccio una linea tratteggiata. Il grafico è concorde
\end{enumerate}
\end{description}
\end{enumerate}
\section{Disequazioni intere e soluzioni}
\begin{esempiot}{$\Delta>0$ $a>0$ e soluzioni}{}
	Troviamo il segno di $2x^2+5x+3$
	\begin{align*}
&2x^2+5x+3=0
\intertext{che risolta}
&x_{1,2}=\dfrac{-5\pm\sqrt{25-24}}{4}=\dfrac{-5\pm 1}{4}=\begin{cases}
x_1=-\dfrac{7}{2}\\
\\x_2=-1
\end{cases}
	\end{align*}
Dato che il delta è positivo e $a$ è positivo il grafico è di tipo DICE quindi è la  figura~\vref{fig:DeltaMaggioreZeroGraficoEsempio7}.	 Al  trinomio dell'esempio possiamo associare quattro disequazioni con quattro soluzioni diverse. 
\begin{align*}
&2x^2+5x+3\geq0&&x\leq-\dfrac{3}{2}\quad x\geq-1\\
&2x^2+5x+3>0&&x<-\dfrac{3}{2}\quad x>-1\\
&2x^2+5x+3\leq 0&&-\dfrac{3}{2}\leq x\leq-1\\
&2x^2+5x+3<0&&-\dfrac{3}{2}< x<-1
\end{align*}
\end{esempiot}
\begin{figure}
	\centering
	\includestandalone[width=8.5cm]{quarto/DisSecGrado/parabolaDeltapiuApiuGrafico2}
	\caption{Segno $\Delta>0$ $a>0$}
		\label{fig:DeltaMaggioreZeroGraficoEsempio7}
\end{figure}
\begin{esempiot}{$\Delta>0$ $a<0$ e soluzioni}{}
	Troviamo il segno di $-3x^2+4x+4$
\begin{align*}
&-3x^2+4x+4=0
\intertext{che risolta}
&x_{1,2}=\dfrac{-4\pm\sqrt{16+48}}{-6}=\dfrac{-4\pm 8}{-6}=\begin{cases}
x_1=-\dfrac{2}{3}\\
\\x_2=2
\end{cases}
\end{align*}
Dato che il delta è positivo e $a$ è negativo il grafico è di tipo DICE quindi è la  figura~\vref{fig:DeltaMaggioreZeroGraficoEsempio8}.	 Al  trinomio dell'esempio possiamo associare quattro disequazioni con quattro soluzioni diverse. 
\begin{align*}
&-3x^2+4x+4\geq0&&-\dfrac{2}{3}\leq x\leq 2\\
&-3x^2+4x+4>0&&-\dfrac{2}{3}< x<2\\
&-3x^2+4x+4\leq 0&&x\leq-\dfrac{2}{3}\quad x\geq 2\\
&-3x^2+4x+4<0&&x<-\dfrac{3}{2}\quad x>2
\end{align*}
\end{esempiot}
\begin{figure}
	\centering
	\includestandalone[width=8.5cm]{quarto/DisSecGrado/parabolaDeltapiuAmenoGrafico2}
	\caption{Segno $\Delta>0$ $a<0$}
	\label{fig:DeltaMaggioreZeroGraficoEsempio8}
\end{figure}
\begin{esempiot}{$\Delta=0$ $a>0$}{}
		Troviamo il segno di $x^2+6x+9$
	\begin{align*}
	&x^2+6x+9=0
	\intertext{che risolta}
	&x_{1,2}=\dfrac{-6\pm\sqrt{36-36}}{2}=-3
	\end{align*}
Delta è uguale a zero il coefficiente $a$ è positivo quindi il grafico è concorde con $a$ ed è come quello della figura~\vref{fig:DeltaUgualeaZeroGraficoEsempio9}
 Al  trinomio dell'esempio possiamo associare quattro disequazioni con quattro soluzioni diverse. 
\begin{align*}
&x^2+6x+9\geq0&&\text{Sempre verificata}\\
&x^2+6x+9>0&&x\neq -3\\
&x^2+6x+9\leq 0&&x=-3\\
&x^2+6x+9<0&&\text{Mai verificata}
\end{align*}
\end{esempiot}
\begin{figure}
	\centering
	\includestandalone[width=8.5cm]{quarto/DisSecGrado/parabolaDeltazeroApiuGrafico2}
	\caption{Segno $\Delta=0$ $a>0$}
	\label{fig:DeltaUgualeaZeroGraficoEsempio9}
\end{figure}
\begin{esempiot}{$\Delta=0$ $a<0$}
	Troviamo il segno di $-4x^2+12x-9$
	\begin{align*}
	&-4x^2+12x-9=0
	\intertext{che risolta}
	&x_{1,2}=\dfrac{-12\pm\sqrt{144-144}}{-8}=\dfrac{3}{4}
	\end{align*}
	Delta è uguale a zero il coefficiente $a$ è negativo quindi il grafico, concorde, è come quello della figura~\vref{fig:DeltaUgualeaZeroGraficoEsempio10}
	Al  trinomio dell'esempio possiamo associare quattro disequazioni con quattro soluzioni diverse. 
	\begin{align*}
	&-4x^2+12x-9\geq0&&x=\dfrac{3}{4}\\
	&-4x^2+12x-9>0&&\text{Mai verificata}\\
	&-4x^2+12x-9\leq 0&&\text{Sempre verificata}\\
	&-4x^2+12x-9<0&&x\neq \dfrac{3}{4}
	\end{align*}
\end{esempiot}
\begin{figure}
	\centering
	\includestandalone[width=8.5cm]{quarto/DisSecGrado/parabolaDeltazeroAmenoGrafico2}
	\caption{Segno $\Delta=0$ $a<0$}
	\label{fig:DeltaUgualeaZeroGraficoEsempio10}
\end{figure}
\begin{esempiot}{$\Delta<0$ $a>0$}
	Troviamo il segno di $x^2+2x+5$
	\begin{align*}
	&x^2+2x+5=0
	\intertext{che risolta}
	&x_{1,2}=\dfrac{-2\pm\sqrt{4-20}}{2}
	\intertext{non ha soluzione}
	\end{align*}
	Delta è minore di zero, il coefficiente $a$ è maggiore di zero quindi il grafico, concorde, è come quello della figura~\vref{fig:DeltaMinoreZeroGraficoEsempio11}
	Al  trinomio dell'esempio possiamo associare quattro disequazioni con quattro soluzioni diverse. 
	\begin{align*}
	&x^2+2x+5\geq0&&\text{Sempre maggiore di zero, mai uguale a zero}\\
	&x^2+2x+5>0&&\text{Sempre maggiore di zero}\\
	&x^2+2x+5\leq 0&&\text{Mai verificata}\\
	&x^2+2x+5<0&&\text{Mai verificata}
	\end{align*}
\end{esempiot}
\begin{figure}
	\centering
	\includestandalone[width=8.5cm]{quarto/DisSecGrado/parabolaDeltamenoApiuGrafico2}
	\caption{Segno $\Delta<0$ $a>0$}
	\label{fig:DeltaMinoreZeroGraficoEsempio11}
\end{figure}
\begin{esempiot}{$\Delta<0$ $a<0$}
	Troviamo il segno di $-2x^2+3x-3$
	\begin{align*}
	&-2x^2+3x-3=0
	\intertext{che risolta}
	&x_{1,2}=\dfrac{-3\pm\sqrt{9-24}}{-4}
	\intertext{non ha soluzione}
	\end{align*}
	Delta è minore di zero, il coefficiente $a$ è minore di zero quindi il grafico, concorde, è come quello della figura~\vref{fig:DeltaMinoreZeroGraficoEsempio12}
	Al  trinomio dell'esempio possiamo associare quattro disequazioni con quattro soluzioni diverse. 
	\begin{align*}
	&-2x^2+3x-3\geq0&&\text{Mai verificata}\\
	&-2x^2+3x-3>0&&\text{Mai verificata}\\
	&-2x^2+3x-3\leq 0&&\text{Sempre minore di zero, uguale a zero mai}\\
	&-2x^2+3x-3<0&&\text{Sempre verificata}
	\end{align*}
\end{esempiot}
\begin{figure}
	\centering
	\includestandalone[width=8.5cm]{quarto/DisSecGrado/parabolaDeltamenoAmenoGrafico2}
	\caption{Segno $\Delta<0$ $a<0$}
	\label{fig:DeltaMinoreZeroGraficoEsempio12}
\end{figure}
%\altapriorita{aggiungere disequazioni frazionarie di secondo grado}
% % % % % % % % % % % % % % % % % % % % % % % % % % % % %
%\section{Metodo grafico}
%\label{sec:MetodoGrafico}
%\begin{figure}
%	
%		\begin{subfigure}[b]{.5\linewidth}
%		\centering
%		\begin{tikzpicture}[line cap=round,line join=round,>=triangle 45,x=1.0cm,y=1.0cm]
%		\draw[->,color=black] (-3,0) -- (3,0);
%		%\foreach \x in {-2.5,-2,-1.5,-1,-0.5,0.5,1,1.5,2,2.5,3}
%		%\draw[shift={(\x,0)},color=black] (0pt,-2pt);
%		\clip(-3,-2.28) rectangle (3,1);
%		\draw [samples=50,rotate around={0:(0,-2.13)},xshift=0cm,yshift=-2.13cm] %plot (\x,\x^2/2/0.2599999999999998);
%		plot(\x,{(\x)^2-0.1}); 
%		%\draw [samples=50,rotate around={0:(0,-2.13)},xshift=0cm,yshift=-2.13cm] plot (\x,(\x)^2/2/0.2599999999999998);
%		\draw (-3,0.37) node[anchor=north west] {$+++++$};
%		\draw (-1.17,0.05) node[anchor=north west] {$x_1$};
%		\draw (1.11,0.05) node[anchor=north west] {$x_2$};
%		\draw (1.17,0.37) node[anchor=north west] {$++++++$};
%		\draw (-0.7,-0.06) node[anchor=north west] {$-----$};
%		\end{tikzpicture}
%		\caption{$\Delta>0$ $a>0$}\label{graf:dis2GDeltaMagZGa1}
%	\end{subfigure}%
%	\begin{subfigure}[b]{.5\linewidth}
%		\centering
%			\begin{tikzpicture}[line cap=round,line join=round,>=triangle 45,x=1.0cm,y=1.0cm]
%			\draw[->,color=black] (-3,0) -- (3,0);
%			%\foreach \x in {-3,-2.5,-2,-1.5,-1,-0.5,0.5,1,1.5,2,2.5}
%			%\draw[shift={(\x,0)},color=black] (0pt,-2pt);
%			\clip(-3,-1) rectangle (3,2.26);
%			\draw [samples=50,rotate around={-180:(0,2.13)},xshift=0cm,yshift=2.13cm] 
%			%plot (\x,\x^2/2/0.2599999999999998);
%			plot(\x,{(-\x)^2-0.1}); 
%			\draw (-0.8,0.37) node[anchor=north west] {$+++++$};
%			\draw (-1.17,0.04) node[anchor=north west] {$x_1$};
%			\draw (1.11,0.06) node[anchor=north west] {$x_2$};
%			\draw (-3,-0.37) node[anchor=north west] {$-----$};
%			\draw (1.39,-0.37) node[anchor=north west] {$-----$};
%			\end{tikzpicture}
%		\caption{$\Delta>0$ $a<0$}\label{graf:dis2GDeltaMagZGb1}
%	\end{subfigure}
%		\begin{subfigure}[b]{.5\linewidth}
%			\centering
%			\begin{tikzpicture}[line cap=round,line join=round,>=triangle 45,x=1.0cm,y=1.0cm]
%			\draw[->,color=black] (-3,0) -- (3,0);
%			%\foreach \x in {-3,-2.5,-2,-1.5,-1,-0.5,0.5,1,1.5,2,2.5}
%			%\draw[shift={(\x,0)},color=black] (0pt,-2pt);
%			\clip(-3,-0.3) rectangle (3,0.5);
%			\draw (-1,0.5)-- (-1,0);
%			\draw (1,0.5)-- (1,0);
%			\draw [line width=1.2pt,dash pattern=on 5pt off 5pt] (-1,0.5) -- (1,0.5);
%			\draw (1,0.5)-- (3,0.5);
%			\draw (-1,0.5)-- (-3,0.5);
%			\draw (-1.02,0.02) node[anchor=north west] {$x_1$};
%			\draw (0.98,0.02) node[anchor=north west] {$x_2$};
%			\end{tikzpicture}
%			\caption{$\Delta>0$ $a>0$}\label{graf:dis2GDeltaMagZGa2}
%		\end{subfigure}%
%		\begin{subfigure}[b]{.5\linewidth}
%			\centering
%				\begin{tikzpicture}[line cap=round,line join=round,>=triangle 45,x=1.0cm,y=1.0cm]
%				\draw[->,color=black] (-3,0) -- (3,0);
%				%\foreach \x in {-3,-2.5,-2,-1.5,-1,-0.5,0.5,1,1.5,2,2.5}
%				%\draw[shift={(\x,0)},color=black] (0pt,-2pt);
%				\clip(-3,-0.3) rectangle (3,0.5);
%				\draw (-1,0.5)-- (-1,0);
%				\draw (1,0.5)-- (1,0);
%				\draw (-1,0.5)-- (1,0.5);
%				\draw [dash pattern=on 5pt off 5pt] (1,0.5)-- (3.0,0.5);
%				\draw [dash pattern=on 5pt off 5pt] (-1,0.5)-- (-3.0,0.5);
%				\draw (-1.02,0.02) node[anchor=north west] {$x_1$};
%				\draw (0.98,0.02) node[anchor=north west] {$x_2$};
%				\end{tikzpicture}
%			\caption{$\Delta>0$ $a<0$}\label{graf:dis2GDeltaMagZGb2}
%		\end{subfigure}
%	\caption{$\Delta>0$}%
%	\label{fig:deltamz}
%\end{figure}
%
%\begin{figure}
%	\begin{subfigure}[b]{.5\linewidth}
%		\centering
%			\begin{tikzpicture}[line cap=round,line join=round,>=triangle 45,x=1.0cm,y=1.0cm]
%			\draw[->,color=black] (-3,0) -- (2.98,0);
%			%\foreach \x in {-3,-2.5,-2,-1.5,-1,-0.5,0.5,1,1.5,2,2.5}
%			%\draw[shift={(\x,0)},color=black] (0pt,-2pt);
%			\clip(-3,-0.29) rectangle (2.98,2);
%			\draw [samples=50,rotate around={0:(0,0)},xshift=0cm,yshift=0cm]
%			%plot (\x,\x^2/2/1.0);
%			plot(\x,{(\x)^2}); 
%			\draw (0,0.02) node[anchor=north west] {$x_1$};
%			\draw (-3,0.37) node[anchor=north west] {$+++++++++++++++++++++$};
%			\end{tikzpicture}
%		\caption{$\Delta=0$ $a>0$}\label{graf:dis2GDeltaUguaZGa1}
%	\end{subfigure}%
%	\qquad
%	\begin{subfigure}[b]{.5\linewidth}
%		\centering
%		\begin{tikzpicture}[line cap=round,line join=round,>=triangle 45,x=1.0cm,y=1.0cm]
%		\draw[->,color=black] (-3,0) -- (2.98,0);
%		%\foreach \x in {-3,-2.5,-2,-1.5,-1,-0.5,0.5,1,1.5,2,2.5}
%		%\draw[shift={(\x,0)},color=black] (0pt,-2pt);
%		\clip(-3,-2) rectangle (2.98,0.29);
%		\draw [samples=50,rotate around={-180:(0,0)},xshift=0cm,yshift=0cm] 
%		%plot (\x,\x^2/2/1.0);
%		plot(\x,{(-\x)^2}); 
%		\draw (0,0.02) node[anchor=north west] {$x_1$};
%		\draw (-3,0.37) node[anchor=north west] {$---------------------$};
%		\end{tikzpicture}
%		\caption{$\Delta=0$ $a<0$}\label{graf:dis2GDeltaDeltaUguaZGb1}
%	\end{subfigure}
%		\begin{subfigure}[b]{.5\linewidth}
%			\centering
%		\begin{tikzpicture}[line cap=round,line join=round,>=triangle 45,x=1.0cm,y=1.0cm]
%		\draw[->,color=black] (-3,0) -- (3,0);
%		%\foreach \x in {-2.5,-2,-1.5,-1,-0.5,0.5,1,1.5,2,2.5,3}
%		%\draw[shift={(\x,0)},color=black] (0pt,-2pt);
%		\clip(-3,-0.3) rectangle (3,0.5);
%		\draw (0,0.5)-- (0,0);
%		\draw [domain=-3:3] plot(\x,{(--0.56-0*\x)/1.12});
%		\draw (-0.03,0.02) node[anchor=north west] {$x_1$};
%		\end{tikzpicture}
%			\caption{$\Delta=0$ $a>0$}\label{graf:dis2GDeltaUguaZGa2}
%		\end{subfigure}%
%		\qquad
%		\begin{subfigure}[b]{.5\linewidth}
%			\centering
%			\begin{tikzpicture}[line cap=round,line join=round,>=triangle 45,x=1.0cm,y=1.0cm]
%			\draw[->,color=black] (-3,0) -- (3,0);
%			\foreach \x in {-2.5,-2,-1.5,-1,-0.5,0.5,1,1.5,2,2.5,3}
%			\draw[shift={(\x,0)},color=black] (0pt,-2pt);
%			\clip(-3,-0.3) rectangle (3,0.5);
%			\draw (0,0.5)-- (0,0);
%			%\draw [domain=-3:3] plot(\x,{(--0.56-0*\x)/1.12});
%			%\draw [dash pattern=on 5pt off 5pt,domain=-3:3] plot(\x,{(--1.18-0*\x)/1.18});
%			\draw [dash pattern=on 5pt off 5pt,domain=-3:3] plot(\x,{(--0.56-0*\x)/1.12});
%			\draw (-0.03,0.02) node[anchor=north west] {$x_1$};
%			\end{tikzpicture}
%			\caption{$\Delta=0$ $a<0$}\label{graf:dis2GDeltaUguaZGb2}
%		\end{subfigure}
%		\caption{$\Delta=0$}%
%		\label{fig:deltaugz}
%\end{figure}
%
%
%\begin{figure}
%	\begin{subfigure}[b]{.5\linewidth}
%		\centering
%		\begin{tikzpicture}[line cap=round,line join=round,>=triangle 45,x=1.0cm,y=1.0cm]
%		\draw[->,color=black] (-3,0) -- (3,0);
%		%\foreach \x in {-3,-2.5,-2,-1.5,-1,-0.5,0.5,1,1.5,2,2.5}
%		%\draw[shift={(\x,0)},color=black] (0pt,-2pt);
%		\clip(-3,-0.3) rectangle (3,2);
%		\draw [samples=50,rotate around={0:(0,0.15)},xshift=0cm,yshift=0.15cm] 
%		plot(\x,{(\x)^2+.2}); 
%		%plot (\x,\x^2/2/0.3);
%		\draw (-3,-0.01) node[anchor=north west] {$+++++++++++++++++++++$};
%		\end{tikzpicture}
%		\caption{$\Delta<0$ $a>0$}\label{graf:dis2GDeltaMinorZGa1}
%	\end{subfigure}%
%	\qquad
%	\begin{subfigure}[b]{.5\linewidth}
%		\centering
%		\begin{tikzpicture}[line cap=round,line join=round,>=triangle 45,x=1.0cm,y=1.0cm]
%		\draw[->,color=black] (-3,0) -- (3,0);
%		%\foreach \x in {-3,-2.5,-2,-1.5,-1,-0.5,0.5,1,1.5,2,2.5}
%		%\draw[shift={(\x,0)},color=black] (0pt,-2pt);
%		\clip(-3,-2) rectangle (3,0.3);
%		\draw [samples=50,rotate around={-180:(0,-0.15)},xshift=0cm,yshift=-0.15cm] 
%		%plot (\x,\x^2/2/0.3);
%		plot(\x,{(-\x)^2+.2}); 
%		\draw (-3,0.37) node[anchor=north west] {$---------------------$};
%		\end{tikzpicture}
%		\caption{$\Delta<0$ $a<0$}\label{graf:dis2GDeltaMinorZGb1}
%	\end{subfigure}
%	\begin{subfigure}[b]{.5\linewidth}
%			\centering
%			\begin{tikzpicture}[line cap=round,line join=round,>=triangle 45,x=1.0cm,y=1.0cm]
%			\draw[->,color=black] (-3,0) -- (3,0);
%			%\foreach \x in {-2.5,-2,-1.5,-1,-0.5,0.5,1,1.5,2,2.5,3}
%			%\draw[shift={(\x,0)},color=black] (0pt,-2pt);
%			\clip(-3,-0.3) rectangle (3,0.5);
%			\draw [domain=-3:3] plot(\x,{(--0.56-0*\x)/1.12});
%			\end{tikzpicture}
%			\caption{$\Delta<0$ $a>0$}\label{graf:dis2GDeltaMinorZGa2}
%	\end{subfigure}%
%	\begin{subfigure}[b]{.5\linewidth}
%			\centering
%			\begin{tikzpicture}[line cap=round,line join=round,>=triangle 45,x=1.0cm,y=1.0cm]
%			\draw[->,color=black] (-3,0) -- (3,0);
%			%\foreach \x in {-2.5,-2,-1.5,-1,-0.5,0.5,1,1.5,2,2.5,3}
%			%\draw[shift={(\x,0)},color=black] (0pt,-2pt);
%			\clip(-3,-0.3) rectangle (3,0.5);
%			\draw [dash pattern=on 5pt off 5pt,domain=-3:3] plot(\x,{(--0.56-0*\x)/1.12});
%			\end{tikzpicture}
%			\caption{$\Delta<0$ $a<0$}\label{graf:dis2GDeltaMinorZGb2}
%	\end{subfigure}
%	\caption{$\Delta<0$}%
%	\label{fig:deltaminz}
%\end{figure}

%\begin{table}[H]
%	%\ContinuedFloat
%	\centering%
%	\subfloat[][$\Delta<0$ $a>0$\label{graf:dis2GDeltaMinorZGa1}]{
%		\begin{tikzpicture}[line cap=round,line join=round,>=triangle 45,x=1.0cm,y=1.0cm]
%		\draw[->,color=black] (-3,0) -- (3,0);
%		%\foreach \x in {-3,-2.5,-2,-1.5,-1,-0.5,0.5,1,1.5,2,2.5}
%		%\draw[shift={(\x,0)},color=black] (0pt,-2pt);
%		\clip(-3,-0.3) rectangle (3,2);
%		\draw [samples=50,rotate around={0:(0,0.15)},xshift=0cm,yshift=0.15cm] 
%		plot(\x,{(\x)^2+.2}); 
%		%plot (\x,\x^2/2/0.3);
%		\draw (-3,-0.01) node[anchor=north west] {$+++++++++++++++++++++$};
%		\end{tikzpicture}
%	}
%	\subfloat[][$\Delta<0$ $a<0$\label{graf:dis2GDeltaMinorZGb1}]{
%		\begin{tikzpicture}[line cap=round,line join=round,>=triangle 45,x=1.0cm,y=1.0cm]
%		\draw[->,color=black] (-3,0) -- (3,0);
%		%\foreach \x in {-3,-2.5,-2,-1.5,-1,-0.5,0.5,1,1.5,2,2.5}
%		%\draw[shift={(\x,0)},color=black] (0pt,-2pt);
%		\clip(-3,-2) rectangle (3,0.3);
%		\draw [samples=50,rotate around={-180:(0,-0.15)},xshift=0cm,yshift=-0.15cm] 
%		%plot (\x,\x^2/2/0.3);
%		plot(\x,{(-\x)^2+.2}); 
%		\draw (-3,0.37) node[anchor=north west] {$---------------------$};
%		\end{tikzpicture}
%	}\quad%
%	\subfloat[][$\Delta<0$ $a>0$\label{graf:dis2GDeltaMinorZGa2}]{
%		\begin{tikzpicture}[line cap=round,line join=round,>=triangle 45,x=1.0cm,y=1.0cm]
%		\draw[->,color=black] (-3,0) -- (3,0);
%		%\foreach \x in {-2.5,-2,-1.5,-1,-0.5,0.5,1,1.5,2,2.5,3}
%		%\draw[shift={(\x,0)},color=black] (0pt,-2pt);
%		\clip(-3,-0.3) rectangle (3,0.5);
%		\draw [domain=-3:3] plot(\x,{(--0.56-0*\x)/1.12});
%		\end{tikzpicture}
%	}
%	\subfloat[][$\Delta<0$ $a<0$\label{graf:dis2GDeltaMinorZGb2}]{
%		\begin{tikzpicture}[line cap=round,line join=round,>=triangle 45,x=1.0cm,y=1.0cm]
%		\draw[->,color=black] (-3,0) -- (3,0);
%		%\foreach \x in {-2.5,-2,-1.5,-1,-0.5,0.5,1,1.5,2,2.5,3}
%		%\draw[shift={(\x,0)},color=black] (0pt,-2pt);
%		\clip(-3,-0.3) rectangle (3,0.5);
%		\draw [dash pattern=on 5pt off 5pt,domain=-3:3] plot(\x,{(--0.56-0*\x)/1.12});
%		\end{tikzpicture}}
%	\caption{$\Delta<0$}%
%	\label{fig:deltaminz}
%	%\label{graf:dis2Ggrafici}
%\end{table}

